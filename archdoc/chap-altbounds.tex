\chapter{Potential revised bound encoding}

Here we describe some possible improvements to bounds encoding described in \cref{sec:bounds}.
In particular we seek to get a more efficient encoding (greater precision, fewer unusable encodings), and to reduce the complexity of the set bounds operation without using any more bits or excessive hardware complexity.
The proposed encoding is a minor alteration to the existing one based on two observations:
\begin{enumerate}
\item That incrementing $T$ in the set bounds operation would not be necessary if the lower $e$ bits of top were decoded as ones, instead of zeros.
\item That zero length capabilities are of little use, and potentially even harmful (see \cref{sec:zerolengthcaps})
\end{enumerate}
As such we consider revising the existing bounds decoding as follows (note ones instead of zeros in low bits of top):
\begin{center}
{
\renewcommand{\arraystretch}{1.5}
\begin{tabular}{r|c|c|c|}
\cline{2-4}
address, $a =$ & $a_\text{top} = a[31:e+9]$ & $a_\text{mid} = a[e+8:e]$  & $a_\text{low} = a[e-1:0]$ \\ \cline{2-4}
base, $b =$    & $a_\text{top}+c_\text{b}$   & $B $ & $0$ \\ \cline{2-4}
top, $t =$     & $a_\text{top}+c_\text{t}$   & $T $ & $1\dots{}1$ \\ \cline{2-4}
\end{tabular}
}
\end{center}
and further redefine the decoded top to be an \emph{inclusive} bound instead of exclusive.
The corrections $c_b$ and $c_t$ remain the same.
This has a number of consequences:
\begin{itemize}
    \item The set bounds implementation no longer has to increment the value of $T$ if the requested top is not exactly represented because the decoding naturally rounds up as desired. 
    This may slightly simplify the implementation.
    \item The smallest length encodable for a given $e$ is $2^e$ (when $B=T$). Zero length capabilities are no longer supported.
    \item The largest length encodable for a given $e$ is $2^{e+9}$ (when $T-B=2^9-1$).
    However in this case the representable range is equal to the dereferenceable range, so it may be necessary to limit the maximum value of $T-B$ to $2^9 - 2$.
    This would ensure that `one past the end' remains within representable range, but would make the maximum \emph{usable} length $511 * 2^e$, which is the same as the current encoding.
    \item Bounds can be set to the entire address space using $B=0$, $T=2^9-1$, $e=23$. 
    This means the maximum exponent is smaller by one and therefore the worst case granularity is now $2^{23}$, or 8 MiB instead of 16 MiB.
\end{itemize}
To retain software compatibility we do not propose to change the architectural definition of \ctop{}, therefore \insnriscvref{CGetLen} would have to take account of the inclusive top value (possibly by adding one to the result) and \insnriscvref{CSetBounds} would have to subtract one from the requested length prior to encoding.
Other instructions will have to adjust bounds checks accordingly.

We have not yet fully evaluated this encoding to see if it is an overall improvement, but include it here for consideration.