\chapter{ABI}
\label{chap:abi}

\cherimcu{} is a hardware-software co-design project, where the ISA and ABI have been carefully designed together to provide the desired compartment model and security guarantees.

\section{Compartment layout}

Each compartment has two reachable regions, bounded by \PCC{} and \CGP{}.
The \PCC{} region contains the compartment's code, read-only data, and \textit{import table}.
Read-only data includes relocation read-only data, which is initialized by the loader at boot time.
A compartment's import table is a read-only table containing capabilities that are used for cross-compartment and cross-library calls, as well as any imported data.

A compartment also has an export table associated with it.
The export table defines the set of functions that are exported from the compartment (callable by others).
This is read by the compartment switcher (see \cref{sec:cross-compart-abi}).

Shared libraries are identical to compartments in structure, except that they lack a \CGP{}.

\section{Access to globals}

Read-only globals are accessed using PCC-relative addressing.
CHERI RISC-V extends the RISC-V \insnref{auipc} (add upper immediate to program counter) instruction to \insnref{auipcc} (add upper immediate to program counter \textit{capability}).
This adds a 20-bit immediate, left shifted by 12, to the current \PCC{} value, giving an address that is within the immediate range of a RISC-V load or store instruction of the target.

Unfortunately, the result of the \insnref{auipcc} instruction may be out of the bounds of \PCC{}.
This does not matter on CHERI systems with 128-bit capabilities because the encoding guarantees that capabilities remain valid 4096 bytes out of bounds.
However, this is not the case with our 64-bit capability encoding that has much tighter `representable bounds'.
The tag bit is cleared if the capability is too far out of bounds, we must therefore modify the standard CHERI-RISC-V relocation scheme to avoid taking capabilities out of bounds in the middle of computing an address.

We are able to solve this problem by having at least a 1-bit overlap between the immediate field of the loads and stores (or \insnref{cincoffset} instructions) and the \insnref{aupicc}.
The \insnref{auipcc} instruction and the second instruction must both displace the \PCC{} in the same direction, keeping the intermediate capability in bounds and hence representable.
If the target address is after the current instruction then both values must be positive, otherwise both values must be negative, which is a simple property for the linker to ensure when applying the relocations.
To make this possible we reduce the shift of \insnref{auipcc} by one, meaning \insnref{auipcc} can always produce a value within the $2$ KiB range required.\footnote{An alternative solution would be to increase the size of the immediate on loads and stores.
On RV32E this could be achieved using the register selection bits that are freed by moving from 32 to 16 registers.
Our first prototypes did this but we choose to remain compatible with RV32I by modifying \insnnoref{auipcc}.
}
This does limit the maximum offset for a relocation to less than $2^{31}$ but this is not a problem in practice due to the limited size of compartments.
Any system that needs more than 2 GiB compartments would likely benefit from a 64-bit address space.

Accesses to read-write globals is very similar.
The \CGP{} register is biased by half the size of the combined globals section (\asm{.data}, \asm{.bss}, and so on).
This means that the full immediate range is accessible for displacements.
With a 12-bit immediate, a single compartment can access 4 KiB of globals in a single load or store (or take their address with a \insnref{cincoffset} instruction).
We define a new instruction, \insnref{auicgp}, and a relocation type that uses it to mirror the \PCC-relative addressing mode.

We rely on linker relaxation to optimize both \PCC{} relative and \CGP{} relative relocations.
This means that relocations within $\pm2$ KiB of \PC{} or \CGP{} require only one instruction.
Given the security incentive to keep compartments small we expect relaxation to work well in the common case.
In particular, if a compartment has more than 4 KiB of mutable global state it may be advisable to split it into multiple compartments or use dynamic allocation.

\section{Export table layout}

\begin{figure}
	\centering
	\begin{bytefield}[bitwidth=\textwidth/32,boxformatting=\centering]{32}
		\bitheader{0-31} \\
		\begin{rightwordgroup}{header}
		\wordbox{2}{\PCC{}} \\
		\wordbox{2}{\CGP{}} \\
		\wordbox{1}{error handler offset}
		\end{rightwordgroup} \\
		\begin{rightwordgroup}{entries}%
		\bitbox{16}{entry point offset} & \bitbox{8}{stack size} & \bitbox{3}{\#args} & \bitbox{1}{ie} & \bitbox{1}{id} & \bitbox{3}{0} \\
		\wordbox[]{1}{\vdots} \\[1ex]
		\bitbox{16}{entry point offset} & \bitbox{8}{stack size} & \bitbox{3}{\#args} & \bitbox{1}{ie} & \bitbox{1}{id} & \bitbox{3}{0}
	\end{rightwordgroup} \\
\end{bytefield}

	\caption{\label{fig:exporttable} Compartment export table layout}
\end{figure}

\cref{fig:exporttable} shows the layout of the export table for a compartment.
Each export table starts with a copy of the \PCC{} and \CGP{} for the target compartment.
The next 32-bits is the offset of the compartment's error handler realtive to \PCC{}.\cbase{}, or $-1$ if the compartment does not have an error handler.
If an error occurs the switcher may jump to this as described in \cref{sec:errorhandling}.
After the header, the export table is comprised of one 32-bit entry per exported function.
The first 16 bits of each entry provide the displacement from the start of the compartment's \PCC{} to the entry point.
This limits a compartment to exporting functions in the first 64 KiB of its code section.
Most compartments have significantly under 64 KiB of code, the few that are larger can sort their internal layout to ensure that the exported functions all fit within the start.

The next 8 bits are the minimum amount of stack space that the function requires.
This allows compartments to be defensive against callers that try to invoke them without enough stack space for their prologues.
If a function requires more than 256 bytes of stack space then it can add a dynamic check on the size of \CSP{} after the compartment switch.

The final 8 bits are reserved for flags, described in the following table:

\begin{center}
	\begin{tabular}{r|l}
		Bits & Meaning \\ \hline
		0-2 & Number of argument registers used. \\
		3 & Interrupts enabled \\
		4 & Interrupts disabled 
	\end{tabular}
\end{center}

The compartment switcher is responsible for clearing all registers except for the used argument registers and so must know how many are used.
The compiler fills in this set.
This provides a value from 0 (no arguments) to 7 (all six argument registers used, plus \creg{5} carrying stack arguments.

Exports from compartments must set either the interrupts-enabled or interrupts-disabled bit.
Code running in a different security context always runs with an explicit interrupt status, to make it easier to reason about compartment behavior.
Functions exposed from shared libraries may set neither, in which case the function will be invoked with the caller's interrupt status.

Each export table entry from a compartment is exposed as a symbol of the form \texttt{\_\_export\_\{compartment name\}\_\{function name\}}.
Each export table entry from a library is exposed as a symbol of the form \texttt{\_\_library\_export\_\{library name\}\_\{function name\}}.
Libraries all use the same name in their export symbols because moving a function from one library to another does not involve running the target in a different security context.
\nwfnote{Eh?  The export name has ``\texttt{library name}'' in it.  What am I missing?}
The existence of multiple libraries is purely to improve auditing: libraries (their entry points, called functions, and the contents of their code sections) can be individually tracked, allowing code-signing rules to be driven by specific implementations of individual libraries.
For example, code signing might require a specific FIPS-certified binary of a crypto library, but allow the shared library providing \ccode{memcpy} to be replaced with a more optimised version.

The function name in the export symbol is mangled according to the Itanium C++ ABI rules.
This provides some defense against accidental (non-malicious) type mismatches in the caller and callee.

\section{Import table layout}

The import table is similar to a captable in structure.
\nwfnote{``captable'' appears only here in this document.}
It is the only piece of state reachable from a compartment that is allowed to contain capabilities that point outside of the compartment's \PCC{} and \CGP{} on system start.
This makes it a single place to audit the compartment graph.
The import table is mutable only by the loader.
After the loader finishes it is reachable only by the read-execute \PCC{} for the compartment, not by any capabilities with store permission.
Import table entries, at run time, are one of three things:

\begin{itemize}
	\item Sealed capabilities to export table entries, used for cross-compartment calls.
	\item Sentry capabilities to library functions.
	\item Capabilities to memory-mapped I/O (MMIO) regions.
\end{itemize}

The loader is responsible for initializing these, based on information provided by the compiler and linker.
Prior to the loader running, import table entries for the first two categories contain addresses of the corresponding export table entry.
Import table entries for MMIO regions contain the start address and the size of the region.
This allows a compartment to be granted a subset of an MMIO region, down to access to a single byte (for example, allowing a compartment to poll the `ready' status of a UART but requiring that it performs a call to the compartment that owns the UART to read or write data with it).
A future version will allow read- or write-only access to MMIO regions.

The loader will populate the import table with capabilities.
Each import table entry that is used for cross-compartment calls will contain a sealed capability that has the bounds of the target compartment's export table and whose address points to the correct entry.
This allows the switcher to load the \PCC{} and \CGP{} values from the start and to jump to the correct address.

The first entry in the import table has the (local) symbol name \ccode{.compartment_switcher}.
It is initialized to 0 at static link time and will be initialized by the loader with a sentry capability for jumping to the compartment switcher.

\section{Cross-compartment calls}
\label{sec:cross-compart-abi}

Cross-compartment calls pass their arguments in the same registers as the RV32E ABI (\creg{10}--\creg{15}).
In addition, any stack arguments are passed via \creg{5} (\creg{t0}).
The callee does not have access to the caller's stack other than via these arguments and so cannot use \CSP-relative addressing for on-stack arguments.

The capability loaded from the import table is passed to the switcher in \creg{6} (\creg{t1}).
The last step on the caller side is to jump to the sentry pointed to by the \ccode{.compartment_switcher} symbol.

If a compartment calls a function that it also exports, and that function has the same interrupt status as the caller, then the compiler may insert a direct call and skip the switcher.


\section{Cross-library calls}

Cross-library calls are simple indirect calls via a capability provided in the import table.
The import table entry contains a sentry capability to the target function.
The \cherimcuisa{} has sentries that enable, disable, or inherit the current interrupt status and so cross-library calls can toggle or preserve the interrupt state.
This makes it easy to reason about the current interrupt state using structured programming idioms.
\nwfnote{Should we more explicitly emphasize that we do not expose any additional mechanism for managing IRQ state?  Specifically we \emph{don't} have \texttt{enable\_interrupts()} and friends.}

If a function explicitly changes interrupt state within a compartment then it will be handled as if it were a library function exported from and consumed by the function.
In this case, the symbol in the export table will be local.

\section{Callbacks}

In some situations, one compartment wishes to provide a callback that another compartment can invoke.
In the \cherimcu{} ABI, this callback is represented as the same form of sealed capability that would be loaded from the import table.
Functions used as cross-compartment callbacks are both exported and imported by the compartment that wishes to take their address.
Taking the address of such a function is simply a load from the import table.

As with non-exported functions that change the interrupt status, the symbol in the export table will be local if the function is not also exported as a directly callable function.

\section{Relocations}
\label{sec:relocs}

The relocations added to RISC-V for \cherimcu{} ABI are listed in \cref{tab:relocs}.
As with existing RISC-V, some of these are in two forms because RISC-V loads and stores place their immediate operands in different locations.
The relocation numbers here are the ones used in the current prototype and are expected to change prior to standardization.

\PCC{} or \CGP{} relative relocations consist of a pair of either \insnref{auipcc} or \insnref{auicgp} plus a 12-bit immediate instruction.
In most cases (when the offset is within $\pm2$ KiB) linker relaxation can reduce this to a single instruction.
The \insnref{AUICGP} instruction uses an entire major opcode and is rarely needed because it is uncommon for a compartment to have more than 4 KiB of read-write global data (arguably a large globals section is an indication that a compartment should be split or refactored).
Therefore, in future we could consider alternative relocations that don't require \insnref{auicgp}, such as a three instruction sequence consisting of \asm{lui}, \asm{addi} and \insnref{cincaddr}.
This would require more complex linker relaxations to retain good code size and efficiency and we have not yet attempted it.

\begin{table}
	\begin{center}
		\begin{tabular}{l|l|p{7cm}}
			Relocation & Value & Meaning \\ \hline
			\asm{CHERI_COMPARTMENT_CGPREL_HI}         & 220 & 20-bit, 11-bit shifted \CGP-relative displacement for use in \insnref{auicgp}. \\
			\asm{CHERI_COMPARTMENT_CGPREL_LO_I}       & 221 & 12-bit \CGP-relative displacement for use in I-type instructions. \\
			\asm{CHERI_COMPARTMENT_CGPREL_LO_S}       & 222 & 12-bit \CGP-relative displacement for use in S-type instructions. \\
			\asm{CHERI_COMPARTMENT_PCCREL_HI}         & 223 & 20-bit, 11-bit shifted \PCC-relative displacement for use in \insnref{auipcc}. \\
			\asm{CHERI_COMPARTMENT_PCCREL_LO}         & 224 & 12-bit displacement for use in I-type instructions. The displacement is relative to the \insnref{auipcc} instruction. \\
			\asm{CHERI_COMPARTMENT_SIZE}              & 225 & The size of the referenced symbol, applied to a \insnref{CSetBounds} instruction. \\
		\end{tabular}
	\caption{\label{tab:relocs}The relocations defined for the \cherimcu{} ABI}
	\end{center}
\end{table}

