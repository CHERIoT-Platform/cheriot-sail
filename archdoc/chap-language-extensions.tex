\chapter{C/C++ language and toolchain extensions}
\label{chap:language}

In addition to the existing CHERI C/C++ extensions, we define a small number of additional extensions that are specific to \cherimcu{}.
CHERI C is already able to compile most existing embedded code that we have tried with no modifications.
Embedded code has to tolerate targets with Harvard architectures, different pointer sizes for different types of data, different memory banks, and so on.
In comparison, a CHERI target is far more like a conventional ISA.

We have not had to change the CHERI C model at all for code running within a compartment.
Our extensions are focused on supporting the compartmentalization model.

\section{Specifying compartments}

We have added an attribute to specify the compartment that implements a given function.
This is used in conjunction with the \flag{-cheri-compartment=} flag passed to the compiler, which specifies the compartment in which the current compilation unit will end up.
The compiler will raise an error if the compartment name for the current compilation unit does not match the name of a function that has an implementation.
For example, if a header file contains the following declarations:

\begin{ccodelisting}
__attribute__((cheri_compartment("example"))) void foo(int);
__attribute__((cheri_compartment("other"))) void bar(void);
\end{ccodelisting}

If \ccode{foo} is implemented then the compiler must be invoked with \flag{-cheri-compartment=example}.
Any call to \ccode{bar} will then be treated as a cross-domain call.

This mechanism allows lightweight annotations on functions that are exposed across compilation units.
Software that already supports a DLL-style linkage model may have macros on public functions for using this.
Other software can easily maintain these annotations for CHERI targets with a macro that expands to nothing for non-CHERI targets.

Adding \ccode{__attribute__((cheri_ccallback))} to a function marks it as a cross-compartment callback.
Taking the address of such a function will give a pointer that can be passed across compartments and allow the recipient to recursively invoke this compartment.

\section{Exposing library calls}

Adding \ccode{__attribute__((cheri_libcall))} to a function marks it as a library call.
The compiler will always generate an indirect call (via a sealing capability from the import table) for all library functions, unless the callee and caller are in the same compilation unit and have compatible interrupt states.

All of the functions that the compiler may insert calls to (such as \ccode{memcpy},\ccode{`__cxa_guard_acquire}, and so on) are assumed to implicitly carry this attribute.
The compiler must be able to insert calls to these without knowing the name of the library that provides them and so this attribute provides a single flat namespace for all such functions.

\section{Controlling interrupt state}

The \ccode{cheri_interrupt_state} attribute controls whether, during execution of the function, interrupts are enabled, disabled, or in whatever state there were for the caller.
It takes a single argument, which must be either \ccode{enabled}, \ccode{disabled}, or \ccode{inherited}.
The attribute can also be written as a C+11 / C2x-style attribute.
For example:

\begin{ccodelisting}
// This function runs with interrupts disabled
[[cheri::interrupt_state(disabled)]]
int disabled(void);
// This function runs with interrupts enabled
__attribute__((cheri_interrupt_state(enabled)))
int enabled(void);
// This function runs with whatever interrupt state the caller has.
__attribute__((cheri_interrupt_state(inherit)))
int inherit(void);
\end{ccodelisting}

All functions that are not exposed across compartment boundaries (including library calls) default to \ccode{inherit}.
Cross-compartment calls default to \ccode{enabled} and must be set to either \ccode{enabled} or \ccode{disabled}.

\section{Linking compartments}

The version of LLD used with \cherimcu{} provides a \flag{-compartment} flag for linking compartments.
This is somewhat similar to \flag{-r}, which creates a relocatable object file.
Linking in this mode marks all symbols except for those in the export table as local.
Unlike \flag{-r}, COMDATs are merged when linking a compartment.
In other respects, this is identical to \flag{-r}: relocations are not processed and will be handled in the final link step.
