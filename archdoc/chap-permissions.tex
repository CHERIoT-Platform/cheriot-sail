\chapter{Permission compression rationale}
\label{chap:permissions}

To devise the permission compression scheme we initially observed a number of constraints on the useful permissions combinations:

\vspace{1em}
\begin{tabularx}{\textwidth}{rX}
$MC \rightarrow (LD \lor SD)$   & Capability read / write is not useful without a load or store permission. \\
$LG \rightarrow (MC \land  LD)$ & Load global requires load capability. \\
$LM \rightarrow (MC \land  LD)$ & Load mutable requires load capability. \\
$SL \rightarrow (MC \land  SD)$ & Store local requires store capability. \\
$SR \rightarrow EX$             & Access system registers only applies to executable capabilities. \\
$EX \rightarrow (MC \land LD)$  & Executable capabilities require load capability for global access. \\
$\neg(EX \land SD)$             & Executable capabilities should not be writable, to enhance security. \\
$\neg((SE \lor US \lor U0) \land (LD \lor SD))$ & Memory permissions should be disjoint from sealing permissions due to separate namespaces. \\
\end{tabularx}
\vspace{1em}

If we apply all of these constraints we find there are only 33 useful permission combinations
(excluding the global bit, which we take to be orthogonal).
Eliminating just one of these combinations allows us to encode them using a 5-bit encoding.
We see limited uses for write-only capabilities outside MMIO, so we chose write-only store-local as the least useful combination, leading to the encoding described in \cref{sec:perms}.
This may be reassessed once we have a clearer picture of use cases for write-only capabilities.
