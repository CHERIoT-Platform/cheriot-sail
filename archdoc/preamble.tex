\errorcontextlines 10000
\usepackage{xparse}
\usepackage{xspace}
\usepackage{environ}
\usepackage{suffix}
\usepackage{xpatch}

%\renewcommand{\baselinestretch}{2} % double space for editors
\usepackage[headings]{fullpage}
\usepackage{bitset}
\usepackage{comment}
\usepackage{graphicx}
\usepackage{marginnote}
\usepackage{booktabs}
\usepackage{ifthen}
\usepackage{bytefield}
\usepackage{rotating}
\usepackage{pdflscape}
\usepackage{tabularx}
\usepackage{multirow, bigdelim}
\usepackage{geometry}
\input{binhex}
\makeatletter\@ifclassloaded{standalone}{%
% The svgnames option conflicts with \documentclass[tikz]{standalone}
\usepackage{xcolor}
}{% else
\usepackage[svgnames]{xcolor}
}\makeatother % end of \@ifclassloaded{standalone}
\definecolor{lightgray}{gray}{0.8}
\usepackage{times}
\usepackage{algpseudocode}
\newcommand{\note}[2]{{\color{blue}[ Note: #1 - #2]}}
%%%%
%%%% For releases, uncomment to cause notes to disappear:
%%%%
\renewcommand{\note}[2]{\relax\ifhmode\unskip\fi}
\newcommand{\deprecated}[2]{{\color{grey}[ Note: #1 - #2]}}
\newcommand{\arnote}[1]{\note{#1}{Alex R.}}
\newcommand{\dcnote}[1]{\note{#1}{David C.}}
\newcommand{\jrtcnote}[1]{\note{#1}{Jess C.}}
\newcommand{\jwnote}[1]{\note{#1}{Jon W.}}
\newcommand{\knnote}[1]{\note{#1}{Kyndylan N.}}
\newcommand{\nwfnote}[1]{\note{#1}{nwf}}
\newcommand{\psnote}[1]{\note{#1}{Peter S.}}
\newcommand{\rmnnote}[1]{\note{#1}{Robert N.}}

% \newcommand{\TODO}[1]{{\color{red}TODO #1}}
\newcommand{\TODO}[1]{}

\usepackage{listings}
\usepackage{rotating}
\usepackage{setspace}
\usepackage{enumitem}
\setlist{noitemsep}
\usepackage{amsmath}
\usepackage{amssymb}
\usepackage{makecell}
\usepackage{hyphenat}

\usepackage[utf8]{inputenc}
\usepackage[T1]{fontenc}

\usepackage{tikz}
 \usetikzlibrary{calc}
 \usetikzlibrary{decorations.pathreplacing}
 \usetikzlibrary{fit}
 \usetikzlibrary{matrix}
 \usetikzlibrary{positioning}
 \usetikzlibrary{shapes}
 \usetikzlibrary{patterns}

\newcommand*{\circnum}[2][gray!25]{%
  \protect\tikz[baseline={([yshift=-1.5pt]n.base)}]%
  \protect\node[fill=#1,shape=circle,inner sep=1pt,draw](n){\tiny #2};}
\newlist{inenum}{enumerate*}{1}
\setlist[inenum]{label={\circnum{\arabic*}}}

% Makes complex expansions slightly more readable
\let\ea\expandafter

\usepackage[scaled=0.82]{beramono}

\makeatletter
\newcommand{\makesailcmds@core}[2]{%
  \input{#2/commands.tex}
  \ea\newcommand\csname #1code\endcsname[1]{%
    \csname #1fcl##1execute\endcsname%
    \bigskip%
  }%
  \ea\WithSuffix\ea\newcommand\csname #1code\endcsname*[1]{%
    \csname #1fcl##1execute\endcsname%
  }%
  \ea\newcommand\csname #1valandfun\endcsname[1]{%
    \csname #1##1\endcsname \csname #1fn#1\endcsname%
  }%
}

% We could have sail macro call us back for more formatting flexibility.
%\newcommand{\saildescribe}[2]{
%  \lstinputlisting[language=sail]{#2}
%
%  \hangindent=\parindent #1
%}

% The following macros define how we would like sail code to be documented.
% There is one per category of sail top-level (val spec, typedef, function, function clause etc)
% currently we only use val and fcl.
% They are called by latex generated by sail with
% #1 the latex for any doc-comment from the sail
% #2 a lstinputlisting invocation that
\newcommand{\saildocval}[2]{%
#2%
\par%
\hangindent=\parindent #1%
\medskip%
}
\newcommand{\saildocfcl}[2]{%
#1 #2%
}
\newcommand{\saildocfn}[2]{%
#1 #2%
}
\newcommand{\saildoctype}[2]{%
#1 #2%
}

\newcommand{\@saildoclabelled@capture}[2]{%
  \global\def\@saildoclabelled@name{#1}%
  \global\def\@saildoclabelled@body{#2}%
}

\newcommand{\@saildocfcl@capture}[2]{%
  \global\def\@saildocfcl@doc{#1}%
  \global\def\@saildocfcl@fcl{#2}%
}

\newcommand{\@saildoc@makeerrcmd}[1]{%
  \ea\def\csname #1@error\endcsname{%
    \GenericError{[saildoc] }{Missing definition}{%
      [saildoc] \@backslashchar#1 should have been defined.\MessageBreak%
      Check your Sail version if you re-generated the LaTeX.%
    }{}%
  }%
}

\@saildoc@makeerrcmd{@saildoclabelled@name}
\@saildoc@makeerrcmd{@saildoclabelled@body}
\@saildoc@makeerrcmd{@saildocfcl@doc}
\@saildoc@makeerrcmd{@saildocfcl@fcl}

\newcommand{\@saildoc@makeforbiddenseccmd}[1]{%
  \ea\def\csname @saildoc@#1\ea\endcsname{%
    \GenericError{[saildoc] }{Forbidden command}{%
      [saildoc] \@backslashchar#1 is not allowed.%
    }{}%
  }%
}

% Always use starred variant
\newcommand{\@saildoc@makenestedseccmd}[2]{%
  \ea\def\csname @saildoc@#1\endcsname{%
    \@ifstar{}{}\csname #2\endcsname*%
  }%
}

\@saildoc@makeforbiddenseccmd{part}
\@saildoc@makeforbiddenseccmd{chapter}
\@saildoc@makeforbiddenseccmd{section}
% subsection is special (but still maps to subsubsection); see below
\@saildoc@makenestedseccmd{subsubsection}{paragraph}
\@saildoc@makenestedseccmd{paragraph}{subparagraph}
\@saildoc@makeforbiddenseccmd{subparagraph}

\let\@saildoc@subsection@allowed\@empty
\newcommand{\@saildoc@subsection@allow}[1]{%
  \ea\def\ea\@saildoc@subsection@allowed\ea{%
    \@saildoc@subsection@allowed%
    \@saildoc@subsection@allowed@iter{#1}%
  }%
}
\@saildoc@subsection@allow{Description}
\@saildoc@subsection@allow{Exceptions}
\@saildoc@subsection@allow{Notes}

\newcommand{\@saildoc@subsection@valid}[1]{}

\newcommand{\@saildoc@subsection@invalid}[1]{%
  \GenericError{[saildoc] }{Invalid subsection}{%
    [saildoc] `#1' is not a valid subsection.%
  }{}%
}

\newcommand{\@saildoc@subsection@duplicate}[1]{%
  \GenericError{[saildoc] }{Duplicate subsection}{%
    [saildoc] `#1' is defined more than once.%
  }{}%
}

\newcommand{\@saildoc@subsection@validate}[1]{{%
  \def\@saildoc@subsection@allowed@iter##1{%
    \ifthenelse{\equal{#1}{##1}}{%
      \let\@saildoc@subsection@validate@action\@saildoc@subsection@valid%
    }{%
    }%
  }%
  \ea\ifx\csname @saildoc@subsection@body@#1\endcsname\relax%
    \let\@saildoc@subsection@validate@action\@saildoc@subsection@invalid%
    \@saildoc@subsection@allowed%
  \else%
    \let\@saildoc@subsection@validate@action\@saildoc@subsection@duplicate%
  \fi%
  \@saildoc@subsection@validate@action{#1}%
}}

\NewEnviron{@saildoc@subsection}[1]{%
  \@saildoc@subsection@validate{#1}%
  \ea\ea\ea\global\ea\ea\ea\def\ea\csname @saildoc@subsection@body@#1\ea\endcsname\ea{\BODY}%
}

\newcommand{\@saildoc@subsection@print}[1]{%
  \ea\ifx\csname @saildoc@subsection@body@#1\endcsname\relax%
  \else%
    \ea\ifx\csname @saildoc@subsection@body@#1\endcsname\@empty%
    \else%
      \subsubsection*{#1}%
      \csname @saildoc@subsection@body@#1\endcsname%
    \fi%
  \fi%
}

\newcommand{\@saildoc@subsection@clear}{{%
  \def\@saildoc@subsection@allowed@iter##1{%
    \ea\global\ea\let\csname @saildoc@subsection@body@##1\endcsname\@undefined%
  }%
  \@saildoc@subsection@allowed%
}}

\newcommand{\@saildoc@xpatchcmd@repeat}[3]{%
  \xpatchcmd{#1}{#2}{#3}{\@saildoc@xpatchcmd@repeat{#1}{#2}{#3}}{}%
}

\newcommand{\@saildoc@environ@guard}[2]{%
  % See \makesailcmds for why this space is needed
  #1{#2} %
}

\newcommand{\@saildoc@textbf}[1]{%
  \ifcsname @capperm@\detokenize{#1}\endcsname%
    \csname @capperm@\detokenize{#1}\endcsname%
  \else%
    \textbf{#1}%
  \fi%
}

\newcommand{\makesailcmds}[2]{%
  \makesailcmds@core{#1}{#2}%
  \ea\newcommand\csname #1isarefbody\endcsname[1]{{%
    %
    % Given:
    %
    %   \saildoclabelled{foo}{\saildocfcl{bar}{baz}}
    %
    % we expand to capture foo, and expand the second argument again to capture
    % bar and baz.
    %
    \global\let\@saildoclabelled@name\@saildoclabelled@name@error%
    \global\let\@saildoclabelled@body\@saildoclabelled@body@error%
    \global\let\@saildocfcl@doc\@saildocfcl@doc@error%
    \global\let\@saildocfcl@fcl\@saildocfcl@fcl@error%
    %
    \let\saildoclabelled\@saildoclabelled@capture%
    \let\saildocfcl\@saildocfcl@capture%
    %
    \csname #1code\endcsname*{##1}%
    \@saildoclabelled@body%
    %
    \ea\ea\ea\ifx\ea\ea\ea\relax\ea\detokenize\ea{\@saildocfcl@doc}\relax%
      \GenericWarning{}{#1 Warning: `##1` is not documented}%
    \fi%
    %
    % Now for the fcl body, rewrite:
    %
    %   Foo
    %   \subsection*{Exceptions}
    %   Bar
    %   \subsection*{Notes}
    %   Baz
    %
    % to:
    %
    %   \@saildoc@environ@guard{\begin{@saildoc@subsection}}{Description}
    %   Foo
    %   \end{@saildoc@subsection}
    %   \@saildoc@environ@guard{\begin{@saildoc@subsection}}{Exceptions}
    %   Bar
    %   \end{@saildoc@subsection}
    %   \@saildoc@environ@guard{\begin{@saildoc@subsection}}{Notes}
    %   Baz
    %   \end{@saildoc@subsection}
    %
    % as well as using \@saildoc@subsubsection etc for all the other section
    % commands. We allow the non-starred \subsection too.
    %
    % The extra space inserted by \@saildoc@environ@guard is required to avoid:
    %
    %   \begin{@saildoc@subsection}{Foo}\end{@saildoc@subsection}
    %
    % as the lack of a token before \end confuses environ and makes it split
    % Foo into argument "F" and body "oo". The space gets stripped away so
    % \BODY will be empty.
    %
    \xpretocmd{\@saildocfcl@doc}{\@saildoc@environ@guard{\begin{@saildoc@subsection}}{Description}}{}{}%
    \@saildoc@xpatchcmd@repeat{\@saildocfcl@doc}{\part}{\@saildoc@part}%
    \@saildoc@xpatchcmd@repeat{\@saildocfcl@doc}{\chapter}{\@saildoc@chapter}%
    \@saildoc@xpatchcmd@repeat{\@saildocfcl@doc}{\section}{\@saildoc@section}%
    \@saildoc@xpatchcmd@repeat{\@saildocfcl@doc}{\subsection*}{\end{@saildoc@subsection}\@saildoc@environ@guard{\begin{@saildoc@subsection}}}%
    \@saildoc@xpatchcmd@repeat{\@saildocfcl@doc}{\subsection}{\end{@saildoc@subsection}\@saildoc@environ@guard{\begin{@saildoc@subsection}}}%
    \@saildoc@xpatchcmd@repeat{\@saildocfcl@doc}{\subsubsection}{\@saildoc@subsubsection}%
    \@saildoc@xpatchcmd@repeat{\@saildocfcl@doc}{\paragraph}{\@saildoc@paragraph}%
    \@saildoc@xpatchcmd@repeat{\@saildocfcl@doc}{\subparagraph}{\@saildoc@subparagraph}%
    \xapptocmd{\@saildocfcl@doc}{\end{@saildoc@subsection}}{}{}%
    %
    % We also want to format various special names in our own way, all of which
    % currently use \textbf in the saildoc output.
    %
    \@saildoc@xpatchcmd@repeat{\@saildocfcl@doc}{\textbf}{\@saildoc@textbf}%
    %
    % Now we have the right \begin and \end macros, with the latter directly
    % visible to environ without any expansion, we can capture their contents
    % by expanding again.
    %
    \@saildocfcl@doc%
    %
    % Finally reassemble the documentation in the right order with the Sail in
    % the right place. We use \csuse to avoid having to pre-initialise
    % everything to \@empty.
    %
    % Also add a label so that instruction references from saildoc resolve
    % correctly. This label is not added by the saildoc generator so we insert
    % it manually here using the sail mangling: <prefix>z<insnname>. This is
    % really a valspec mangling, which allows us to link to the description
    % rather than the function body and so saildoc's inability to reference
    % function clauses in markdown turns out to be useful.
    %
    \label{#1z##1}%
    \@saildoc@subsection@print{Description}%
    %
    \subsubsection*{Semantics}%
    \phantomsection%
    \label{\@saildoclabelled@name}%
    \noindent\@saildocfcl@fcl%
    %
    \@saildoc@subsection@print{Exceptions}%
    %
    \@saildoc@subsection@print{Notes}%
    %
    % Reset state for next time
    %
    \@saildoc@subsection@clear%
  }}%
}
\makeatother


\makesailcmds{sailRISCV}{sail_latex_riscv}
\newcommand{\isail}[1]{\lstinline[language=sail]{#1}}
\newcommand{\optype}[1]{\subsection{#1 Instructions}}

% Must be included later than setspace, otherwise all footnote hyperlinks
% point to the title page.
%   PS HACK
%\usepackage[hidelinks]{hyperref}
\usepackage[colorlinks]{hyperref}
% Glossaries must be included after hyperref.
\usepackage[toc,nonumberlist]{glossaries}
\usepackage[nottoc]{tocbibind}
\usepackage[capitalise]{cleveref}
  \Crefname{appendix}{Appendix}{Appendices}
  \Crefname{figure}{Figure}{Figures}
\usepackage{footnote}
\usepackage{threeparttable}
\definecolor{CodeColour}{rgb}{0.9,0.9,0.9} %Light grey
\lstset{basicstyle=\small\ttfamily,
        stringstyle=\textit, %italic strings
        keywordstyle=\textbf, %Bold keywords
        commentstyle=,
        breaklines=true, % Wrap long lines
        numbers=left, % Line numbers on the left
        frame=l, %Border on the left
        framerule=0.8pt, % Thick border
        backgroundcolor=\color{CodeColour}, %Coloured code listings
        numberstyle={\small \oldstylenums},  %tiny, old style line numbers
		%stepnumber=5, % Number every fifth line
        numbersep=5pt, % Five points between the line numbers and the text
        tabsize=4
}
\lstdefinelanguage{llvm}
{
	morekeywords={private, constant, i8, i32, define, icmp, label, i64, call, void, ret, getelementptr, br, load, align, nounwind},
	morekeywords={addrspace, inttoptr, ptrtoint, tail},
	morecomment=[l];
}%

\lstdefinelanguage{sail}
  { morekeywords={val,function,cast,type,forall,foreach,from,to,overload,operator,enum,union,undefined,exit,and,assert,sizeof,
      scattered,register,inc,dec,if,then,else,effect,let,as,@,in,end,Type,Int,Order,match,clause,struct},
    morestring=[b]",
    stringstyle={\ttfamily\color{red}},
    showstringspaces=false,
    morecomment=[l][\itshape\color{DarkGreen}]{//},
    morecomment=[s][\itshape\color{DarkGreen}]{/*}{*/},
    deletestring=[bd]{'},
    escapechar=\#,
    emphstyle={\it},
    numbers=none,
    frame=none,
    backgroundcolor=\color{White},
    aboveskip=0em,
    belowskip=0em,
  }

\lstdefinelanguage{bluespec}
{ morekeywords={function,endfunction,for,struct,typedef,Integer,Bit,Bool,TSub,TAdd,return,if,method},
  morestring=[b]"'=’-<>,
  stringstyle={\ttfamily\color{red}},
  morecomment=[l][\itshape\color{DarkGreen}]{//},
  morecomment=[s][\itshape\color{DarkGreen}]{/*}{*/},
  emphstyle={\it},
}

\lstnewenvironment{ccodelisting}{\lstset{language=C}}{}
\lstnewenvironment{llvmlisting}{\lstset{language={llvm}}}{}
\newcommand{\ccode}[1]{\lstinline[backgroundcolor=\color{white},language=C]|#1|}
\newcommand{\llvmir}[1]{\lstinline[backgroundcolor=\color{white},language={llvm}]|#1|}
\newcommand{\asm}[1]{\lstinline[backgroundcolor=\color{white},language={}]|#1|}
\lstnewenvironment{asmcode}{\lstset{language=}}{}
\newcommand{\regname}[1]{{\small\ttfamily\$#1}}

\newcommand{\baselineboxformatting}[1]{%
  % Measure size of contents
  \sbox0{#1}%
  % Use the difference between the contents' height and the bitbox's height,
  % clamped to [-.44\baselineskip, 0], as our minimum depth.
  \setlength{\skip0}{\ht0 - \height}%
  \ifdim\skip0>0pt%
    \setlength{\skip0}{0}%
  \else%
    \ifdim\skip0<-.44\baselineskip%
      \setlength{\skip0}{-.44\baselineskip}%
    \fi%
  \fi%
  \centering\rule[\skip0]{0pt}{\height}#1%
}
\bytefieldsetup{boxformatting=\baselineboxformatting}

% Well this is gross, but it lets us align baselines between labels and
% bytefields in tabular environments... by pretending that the label is
% a "bytefield" of one bit of the right width, with no bounding lines.
\newcommand{\raiseforbf}[1]{%
  {\begin{bytefield}[bitwidth=\widthof{#1}]{1} \bitbox[]{1}{#1} \end{bytefield}}%
}

\makeatletter
\newdimen\rotateinbitbox@height
\newcommand{\rotateinbitbox}[1]{%
  \rotateinbitbox@height=\height%
  \rotatebox{90}{\makebox[\rotateinbitbox@height][c]{#1}}%
}
\makeatother

\hyphenation{CheriBSD}
\hyphenation{FreeBSD}
\hyphenation{CTSRD}
\hyphenation{CheriRTOS}
\hyphenation{CompartOS}

\newcommand{\cheriot}{CHERIoT}
\newcommand{\cherimcu}{\cheriot{}}
\newcommand{\cherimcuos}{\cheriot{} RTOS}
\newcommand{\cherimcuisa}{\cheriot{} ISA}

\reversemarginpar
\setlength{\marginparwidth}{1.2in}
\let\oldmarginpar\marginpar
\renewcommand\marginpar[1]{\-\oldmarginpar[\raggedright\footnotesize #1]%
{\raggedright\footnotesize #1}}

\newcommand{\defn}[1]{\textbf{#1}}
\newcommand{\flag}[1]{{\tt \small #1}}
\newcommand{\literal}[1]{{\tt \small #1}}
\newcommand{\function}[1]{{\tt \small #1}}
\newcommand{\keyword}[1]{\textit{#1}}

% Register names
\newcommand{\reg}[1]{{\bf R#1}}		% MIPS register numbers
\newcommand{\creg}[1]{{\bf C#1}}	% Capability register numbers
\newcommand{\mreg}[1]{{\bf \$#1}}	% MIPS ABI register names
\newcommand{\PC}{{\bf PC}}
\newcommand{\SP}{{\bf SP}}
\newcommand{\EPC}{{\bf EPC}}
\newcommand{\PCC}{{\bf PCC}}
\newcommand{\CGP}{{\bf CGP}}
\newcommand{\DDC}{{\bf DDC}}
\newcommand{\CNULL}{{\bf CNULL}}
\newcommand{\IDC}{{\bf IDC}}
\newcommand{\TSC}{{\bf TSC}}
\newcommand{\KRC}{{\bf KR1C}}
\newcommand{\KQC}{{\bf KR2C}}
\newcommand{\KCC}{{\bf KCC}}
\newcommand{\KDC}{{\bf KDC}}
\newcommand{\ErrorEPCC}{{\bf ErrorEPCC}}
\newcommand{\EPCC}{{\bf EPCC}}
\newcommand{\CULR}{{\bf CULR}}
\newcommand{\CPLR}{{\bf CPLR}}
\newcommand{\EXL}{{\bf EXL}}
\newcommand{\KSU}{{\bf KSU}}
\newcommand{\ErrorEPC}{{\bf ErrorEPC}}
\newcommand{\causereg}{{\bf cause}}
\newcommand{\capcausereg}{{\bf capcause}}

% RISC-V new register names
\newcommand{\UTCC}{{\bf UTCC}}
\newcommand{\UTDC}{{\bf UTDC}}
\newcommand{\UScratchC}{{\bf UScratchC}}
\newcommand{\UEPCC}{{\bf UEPCC}}
\newcommand{\STCC}{{\bf STCC}}
\newcommand{\STDC}{{\bf STDC}}
\newcommand{\SScratchC}{{\bf SScratchC}}
\newcommand{\SEPCC}{{\bf SEPCC}}
\newcommand{\MTCC}{{\bf MTCC}}
\newcommand{\MTDC}{{\bf MTDC}}
\newcommand{\MScratchC}{{\bf MScratchC}}
\newcommand{\MEPCC}{{\bf MEPCC}}
\newcommand{\xTCC}{{\bf {\it x}TCC}}
\newcommand{\xTDC}{{\bf {\it x}TDC}}
\newcommand{\xScratchC}{{\bf {\it x}ScratchC}}
\newcommand{\xEPCC}{{\bf {\it x}EPCC}}
\newcommand{\xccsr}{\texttt{{\it x}ccsr}}
\newcommand{\mccsr}{\texttt{mccsr}}
\newcommand{\sccsr}{\texttt{sccsr}}
\newcommand{\uccsr}{\texttt{uccsr}}
\newcommand{\mshwm}{\texttt{mshwm}}
\newcommand{\mshwmb}{\texttt{mshwmb}}
% RISC-V existing registers
\newcommand{\xtval}{\texttt{{\it x}tval}}
\newcommand{\xtvec}{\texttt{{\it x}tvec}}
\newcommand{\mtval}{\texttt{mtval}}
\newcommand{\mtvec}{\texttt{mtvec}}
\newcommand{\stvec}{\texttt{stvec}}
\newcommand{\utvec}{\texttt{utvec}}
\newcommand{\xepc}{\texttt{{\it x}epc}}
\newcommand{\mepc}{\texttt{mepc}}
\newcommand{\sepc}{\texttt{sepc}}
\newcommand{\uepc}{\texttt{uepc}}
\newcommand{\xcause}{\texttt{{\it x}cause}}
\newcommand{\mcause}{\texttt{mcause}}
\newcommand{\scause}{\texttt{scause}}
\newcommand{\ucause}{\texttt{ucause}}
\newcommand{\xRET}{\insnnoref{{\it x}RET}}

\newcommand{\AL}{{\bf AL}}
\newcommand{\AX}{{\bf AX}}
\newcommand{\FS}{{\bf FS}}
\newcommand{\GS}{{\bf GS}}
\newcommand{\EAX}{{\bf EAX}}
\newcommand{\CAX}{{\bf CAX}}
\newcommand{\CBP}{{\bf CBP}}
\newcommand{\CBX}{{\bf CBX}}
\newcommand{\CFS}{{\bf CFS}}
\newcommand{\CGS}{{\bf CGS}}
\newcommand{\CDI}{{\bf CDI}}
\newcommand{\CIP}{{\bf CIP}}
\newcommand{\KGS}{{\bf KGS}}
\newcommand{\CRTWO}{{\bf CR2}}
\newcommand{\CRFOUR}{{\bf CR4}}
\newcommand{\CRFIVE}{{\bf CR5}}
\newcommand{\CRTWELVE}{{\bf CR12}}
\newcommand{\CS}{{\bf CS}}
\newcommand{\CSI}{{\bf CSI}}
\newcommand{\CSP}{{\bf CSP}}
\newcommand{\IDT}{{\bf IDT}}
\newcommand{\IST}{{\bf IST}}
\newcommand{\KSC}{{\bf KSC}}
\newcommand{\RAX}{{\bf RAX}}
\newcommand{\RBP}{{\bf RBP}}
\newcommand{\RBX}{{\bf RBX}}
\newcommand{\RCX}{{\bf RCX}}
\newcommand{\RDI}{{\bf RDI}}
\newcommand{\REX}{{\bf REX}}
\newcommand{\RIP}{{\bf RIP}}
\newcommand{\RSI}{{\bf RSI}}
\newcommand{\RSP}{{\bf RSP}}
\newcommand{\RFLAGS}{{\bf RFLAGS}}
\newcommand{\TSS}{{\bf TSS}}
\newcommand{\CSTAR}{{\bf CSTAR}}
\newcommand{\STAR}{{\bf IA32\_STAR}}
\newcommand{\LSTAR}{{\bf IA32\_LSTAR}}
\newcommand{\KGSBASE}{{\bf IA32\_KERNEL\_GS\_BASE}}

% Capability register fields
\newcommand{\ctag}{{\bf tag}}
\newcommand{\csealed}{{\bf s}}
\newcommand{\cperms}{{\bf perms}}
\newcommand{\cuperms}{{\bf uperms}}
\newcommand{\cflags}{{\bf flags}}
\newcommand{\cotype}{{\bf otype}}
\newcommand{\ccursor}{{\bf cursor}}
\newcommand{\caddress}{{\bf address}}
\newcommand{\cbase}{{\bf base}}
\newcommand{\clength}{{\bf length}}
\newcommand{\coffset}{{\bf offset}}
\newcommand{\cbound}{{\bf top}}

%  CHERI-128 v1 capability fields
\newcommand{\ctobase}{{\bf toBase}}
\newcommand{\ctobound}{{\bf toBound}}
\newcommand{\cformat}{{\bf FT}}
\newcommand{\cexponent}{{\bf e}}
\newcommand{\csign}{{\bf SN}}

%  CHERI-128 v1 capability fields
\newcommand{\cbasebits}{{\bf baseBits}}
\newcommand{\ctopbits}{{\bf topBits}}
\newcommand{\ccarries}{{\bf C}}

% CHERI-128 candidate 3 fields
\newcommand{\ctop}{{\bf top}}
%\newcommand{\rbase}{\textbf{base\textsubscript{req}}}
\newcommand{\rbase}{\textbf{base\_req}}
\newcommand{\cbasecorrection}{\textbf{c\textsubscript{b}}}
%\newcommand{\cbasecorrection}{\textbf{c\_b}}
%\newcommand{\rlength}{\textbf{length\textsubscript{req}}}
\newcommand{\rlength}{\textbf{rlength}}
\newcommand{\ctopcorrection}{\textbf{c\textsubscript{t}}}
%\newcommand{\ctopcorrection}{\textbf{ctop}} - SWM: why use this version?
\newcommand{\cB}{{\bf B}}
\newcommand{\cT}{{\bf T}}
\newcommand{\caddr}{{\bf a}}

% Architectural parameters
\newcommand{\xlen}{{\texttt{XLEN}}}
\newcommand{\clen}{{\texttt{CLEN}}}

% Field used in several compression formats
\newcommand{\cmuperms}{$\boldsymbol{\mu}\textbf{perms}$}

% Stylized permission bit.  Starred form omits Permit\_ prefix for informal
% references.
\NewDocumentCommand{\capperm}{sm}{\textsc{\small\IfBooleanTF{#1}{}{Permit\_}#2}\xspace}
% Permission bit convenience macros
% We define a short form for normal use and a long @-command form for internal
% use by saildoc.
\makeatletter
\NewDocumentCommand{\makecapperm}{smm}{%
  \def\@make@capperm##1##2{%
    \ea\NewDocumentCommand\csname @capperm@\detokenize{##2#3}\endcsname{s}{%
      \IfBooleanTF{####1}{\capperm*}{\capperm##1}{#3}%
    }%
    \ea\ea\ea\let\ea\csname capperm#2\ea\endcsname%
      \csname @capperm@\detokenize{##2#3}\endcsname%
  }%
  \IfBooleanTF{#1}{\@make@capperm{*}{}}{\@make@capperm{}{Permit\_}}%
  \let\@make@capperm\undefined%
}
\makeatother
\makecapperm{ASR}{Access\_System\_Registers}
\makecapperm*{ID}{Iterrupt\_Disable}
\makecapperm{Cid}{Set\_CID}
\makecapperm{CInvoke}{CInvoke}
\makecapperm{L}{Load}
\makecapperm{LC}{Load\_Capability}
\makecapperm{LM}{Load\_Mutable}
\makecapperm{S}{Store}
\makecapperm{Seal}{Seal}
\makecapperm{MC}{Load\_Store\_Capability}
\makecapperm{SC}{Store\_Capability}
\makecapperm{SLC}{Store\_Local\_Capability}
\makecapperm{Unseal}{Unseal}
\makecapperm{X}{Execute}
% No Permit_, so always use starred form even if not given
\makecapperm*{G}{Global}
\makecapperm{ILG}{Load\_Global}
\makecapperm*{AL}{Allocator}
\makecapperm*{UZ}{User\_Perm0}

\makeatletter
\newcommand{\@insnlabelname}[2]{insn:#1:#2}
\newcommand{\@insnlabel}[2]{\label{\@insnlabelname{#1}{#2}}}

% If no optional argument is passed (internally, if an empty second argument is
% passed), the lowercased text is the label reference
\newcommand{\@insnrefnofont}[3]{{%
  \if\relax\detokenize{#2}\relax% optional arg not passed
    % NB: \lowercase is not expandable so is outside the \def.
    \lowercase{\def\@insnrefnofont@insnname{#3}}%
  \else% optional arg passed
    \def\@insnrefnofont@insnname{#2}%
  \fi%
  \hyperref[\@insnlabelname{#1}{\@insnrefnofont@insnname}]{#3}%
}}

\NewDocumentCommand{\@insnfmt}{sm}{%
  \IfBooleanTF{#1}{#2}{{\tt \small #2}}%
}

\NewDocumentCommand{\@insnref}{smmm}{%
  \IfBooleanTF{#1}{\@insnfmt*}{\@insnfmt}{\@insnrefnofont{#2}{#3}{#4}}%
}

\newcommand{\@makeinsncmds@explicit}[2]{%
  \ea\newcommand\csname insn#1labelname\endcsname[1]{\@insnlabelname{#2}{##1}}%
  \ea\newcommand\csname insn#1label\endcsname[1]{\@insnlabel{#2}{##1}}%
  \ea\NewDocumentCommand\csname insn#1ref\endcsname{sO{}m}{%
    \IfBooleanTF{##1}{\@insnref*}{\@insnref}{#2}{##2}{##3}%
  }%
}
\newcommand{\@makeinsncmds}[1]{\@makeinsncmds@explicit{#1}{#1}}

\@makeinsncmds{mips}
\@makeinsncmds{riscv}

\newcommand{\definsnarch}[1]{\def\@definsnarch{#1}}
\@makeinsncmds@explicit{}{\@definsnarch}

\let\insnnoref\@insnfmt
\makeatother

% Default is RISC-V
\definsnarch{riscv}

\newcommand{\cherithreeop}[5][NOHEADER]{
\begin{bytefield}{32}
	\ifthenelse{\equal{#1}{NOHEADER}}{}
	{\bitheader[endianness=big]{0,5,6,10,11,15,16,20,21,25,26,31}}\\
	\bitbox{6}{{\color{Grey}0x12}}
	\bitbox{5}{0x0}
	\bitbox{5}{#3}
	\bitbox{5}{#4}
	\bitbox{5}{#5}
	\bitbox{6}{#2}
\end{bytefield}%
}
\newcommand{\cheritwoop}[4][NOHEADER]{\cherithreeop[#1]{{\color{Grey}0x3f}}{#3}{#4}{#2}}
\newcommand{\cherioneop}[3][NOHEADER]{\cheritwoop[#1]{{\color{Grey}0x1f}}{#3}{#2}}


\newcommand{\usesDDCinsteadofNULL}[1]{%
\paragraph{Note:}
If the encoded value of \emph{#1} is zero, this instruction will use
\DDC{} as the \emph{#1} operand
}

% When specifying instructions in pseudocode:
\newcommand{\algorithmicnot}{\textbf{not}}
\newcommand{\algorithmicand}{\textbf{and}}
\newcommand{\algorithmicor}{\textbf{or}}
\newcommand{\algorithmictrue}{\textbf{true}}
\newcommand{\algorithmicfalse}{\textbf{false}}
\newcommand{\algorithmicwith}{\textbf{with}}

\DeclareMathOperator{\msb}{msb}
\DeclareMathOperator{\clz}{clz}

% Markdown and Sail's LaTeX backend don't do well with literal < and >, so add
% \lt and \gt macros like \le and \ge.
\let\lt<
\let\gt>

\makeatletter
\newcount\@autogrid@col
\newcount\@autogrid@cols
\def\@autogrid@cr{%
  \global\advance\@autogrid@col 1\relax%
  \ifnum\@autogrid@col=\@autogrid@cols%
    \def\@autogrid@cr@body{\cr}%
    \global\@autogrid@col=0\relax%
  \else%
    \def\@autogrid@cr@body{&}%
  \fi%
  \@autogrid@cr@body%
  \let\\\@autogrid@cr%
}
\newenvironment{autogrid}[1]{%
  \let\@autogrid@format\@empty%
  \@autogrid@cols=\numexpr(#1)\relax%
  \@autogrid@col=0\relax%
  \loop\ifnum\@autogrid@col<\@autogrid@cols%
    \ea\def\ea\@autogrid@format\ea{\@autogrid@format l}%
    \advance\@autogrid@col 1\relax%
  \repeat%
  \@autogrid@col=0\relax%
  \def\@autogrid@begintabular{\begin{tabular}}%
  \ea\@autogrid@begintabular\ea{\@autogrid@format}%
  \let\\\@autogrid@cr%
}{%
  \end{tabular}%
}
\makeatother


\makeatletter\@ifclassloaded{standalone}{%
% No need for glossary or bibliography when building tikz figures
}{% else
\renewcommand{\glossarypreamble}{\label{glossary}}
\makeglossaries

\newglossaryentry{abstract capability}
{
  name=abstract capability,
  description={
%     Abstract capabilities maintain the appearance of capability
%     lifespan across operations that violate architectural \gls{capability
%     provenance}.
%     For example, abstract capabilities remain valid despite an OS kernel
%     swapping them to and from disk, which requires that any architectural
%     \gls{capability} in the swapped memory have its \gls{capability tag}
%     restored through re-derivation
    Abstract capabilities are a conceptual abstraction that overlays the
    concrete capabilities of the architecture to describe the intended
    maintenance of capability lifespan across operations that violate
    architectural \gls{capability provenance}.
    For example, if an OS kernel
    swaps a page containing a capability to and from disk, 
    it will have to have its \gls{capability tag}
    restored through re-derivation, so there is no longer an
    architectural provenance relationship between the two, but for
    application-level reasoning it is sometimes useful to regard there
    to be one}
}

\newglossaryentry{address}
{
  name=address,
  description={An integer address suitable for dereference within an address
    space.
    In \gls{CHERI-MIPS}, \glspl{capability} are always interpreted in terms of
    \glspl{virtual address}.
    In \gls{CHERI-RISC-V}, \glspl{capability} may be interpreted as
    \glspl{virtual address} -- or \glspl{physical address} when operating in
    Machine Mode}
}

\longnewglossaryentry{capability}
{
  name=capability,
  plural=capabilities,
}
{
  A capability contains an \gls{address}, \gls{capability bounds}
  describing a range of bytes within which addresses may be
  \glslink{dereference}{dereferenced}, \gls{capability permissions}
  controlling the forms of dereference that may be permitted (e.g., load or
  store), a \gls{capability tag} protecting \gls{capability validity}
  (integrity and \gls{capability provenance}), and a \gls{capability object type}
  indicating whether it is a \gls{sealed capability}
  (and, if so, under which \gls{capability object type} they are sealed)
  or \gls{unsealed capability}.
  The address embedded within a capability may be a \gls{virtual address} or
  a \glspl{physical address} depending on the current addressing mode; when
  used to authorize (un)sealing, the address is instead a
  \gls{capability object type}.

  In CHERI, capabilities are used to implement \glspl{pointer} with additional
  protections in aid of \gls{fine-grained memory protection},
  \gls{control-flow robustness}, and other higher-level protection models such
  as \gls{software compartmentalization}.
  Unlike a \gls{fat pointer}, capabilities are subject to
  \gls{capability provenance}, ensuring that they are derived from a prior
  valid capability only via valid manipulations, and \gls{capability
  monotonicity}, which ensures that manipulation can lead only to
  non-increasing rights.
  CHERI capabilities provide strong compatibility with C-language pointers and
  Memory Management Unit (MMU)-based system-software designs, by virtue of
  its \gls{hybrid capability model}.

  Architecturally, a capability can be viewed as an \gls{address} equal to the
  sum of the \gls{capability base} and \gls{capability offset}, as well as
  associated metadata.
%\psnote{Perhaps this base/offset view should now be de-emphasised?  It's
%arguably  implementation detail in any case}
  Dereferencing a capability is done relative to that address.
%  The implementation may choose to store the pre-computed address
%  combining the base and offset, to avoid an implied addition on each memory
%  access, and to similarly store the base and length as pre-computed
%  addresses.
  The size of an in-memory capability may be smaller than the sum of its
  architectural fields (such as base, offset, and permissions) if a
  \gls{compressed capability} mechanism, such as \gls{CHERI Concentrate}, is
  used.

  In the ISA, capabilities may be used explicitly via \gls{capability-based
  instructions}, an application of the \gls{principle of intentional use},
  but also implicitly using \glslink{legacy instructions}{legacy load
  and store instructions} via the \gls{default data capability}, and
  instruction fetch via the \gls{program-counter capability}.
  A capability is either sealed or unsealed, controlling whether it has
  software-defined or instruction-set-defined behavior, and whether or not its
  fields are immutable.

  Capabilities may be held in a \gls{capability register} in a dedicated
  \gls{capability register file}, a \gls{merged register file}, or a
  suitably aligned \gls{tagged memory}.
}

\newglossaryentry{capability base}
{
  name=capability base,
  description={The lower of the two \gls{capability bounds}, from which
    the \gls{address} of a \gls{capability} can be calculated by using
    the \gls{capability offset}}
}

\newglossaryentry{capability bounds}
{
  name=capability bounds,
  description={Upper and lower bounds, associated with each
    \gls{capability}, describing a range of \glspl{address} that may
    be \glslink{dereference}{dereferenced} via the capability.
    Architecturally, bounds are with respect to the \gls{capability base},
    which provides the lower bound, and \gls{capability length}, which
    provides the upper bound when added to the base.
    The bounds may be empty, connoting no right to dereference at any
    address.
    The address of a capability may float outside of the
    dereferenceable bounds; with a \gls{compressed capability}, it may not
    be possible to represent all possible \glslink{out of
      bounds}{out-of-bounds} addresses.
    Bounds may be manipulated subject to \gls{capability monotonicity}
    using \gls{capability-based instructions}}
}

\newglossaryentry{capability length}
{
  name=capability length,
  description={The distance between the lower and upper \gls{capability
    bounds}}
}

\newglossaryentry{capability monotonicity}
{
  name=capability monotonicity,
  description={Capability monotonicity is a property of the instruction set
    that any requested manipulation of a \gls{capability}, whether in a
    \gls{capability register} or in memory, either leads to strictly
    non-increasing rights, clearing of the \gls{capability tag}, or a
    hardware exception.    
    \knnote{I presume that the ``rights'' of a capability are
    determined by its permissions and its bounds, but not by its
    sealedness. In other words, increasing the permissions or bounds
    of a capability would increase its rights, but unsealing a
    capability would not increase its right. If this is correct,
    perhaps it could help to explicitly state this here.}    
    Controlled violation of monotonicity can be achieved via the exception
    delivery mechanism, which grants rights to additional capability
    register, and also by the \gls{CInvoke} instruction, which may
    unseal (and jump to) suitably checked \glspl{sealed
    capability}.
    \knnote{The exception delivery mechanism and the CCall instruction
    do not violate monotonicity, since they do not increase the rights of
    any capability. They do violate the monotonicity of the set of
    reachable rights (see \ref{sec:model-monotonicity}), because an
    exception makes a capability reachable that might not have been
    reachable before (namely the KCC) and the CCall instruction
    unseals capabilities without needing a capability that has the
    authority to unseal them. Perhaps it would be worth creating a
    glossary entry for the set of reachable rights, and mention that
    this set is monotonic as long as no exceptions are raised or
    CCalls are executed.}}
}

\newglossaryentry{capability object type}
{
  name=capability object type,
  description={In addition to \glslink{fat pointer}{fat-pointer} metadata such
    as \gls{capability bounds} and \gls{capability permissions}, \glspl{capability} also contain an integer object type.
    The object type space is partitioned into a range of non-reserved and
    \gls{reserved capability object type} types.
    The \glspl{reserved capability object type} are hardware-interpreted and
    include \glspl{unsealed capability} or \glspl{sealed entry capability}.
    If the object type is one of the non-\glspl{reserved capability object type},
    the capability is a \gls{sealed capability with an object type}.
    For \glspl{sealed capability with an object type}, the object type is set during a
    sealing operation to the \gls{address} of the \gls{sealing capability}.
    Object types can be used to link a sealed \gls{code capability} and a
    sealed \gls{data capability} when used with \gls{CInvoke} to implement a
    software object model or to implement software-defined tokens of authority}
}

\newglossaryentry{capability offset}
{
  name=capability offset,
  description={The distance between \gls{capability base} and the
    \gls{address} accessed when the \gls{capability} is used as a \gls{pointer}}
}

\newglossaryentry{capability permissions}
{
  name=capability permissions,
  description={A bitmask, associated with each \gls{capability},
    describing a set of ISA- or software-defined operations that may be
    performed via the capability.
    ISA-defined permissions include load data, store data, instruction fetch,
    load capability, and store capability.
    Permissions may be manipulated subject to \gls{capability monotonicity}
    using \gls{capability-based instructions}}
}

\newglossaryentry{capability provenance}
{
  name=capability provenance,
  description={
% The property that, following manipulation, a \gls{capability}
%     remains valid for use only if it is derived from another valid capability
%     using a valid capability operation.
%     Provenance is implemented using a \gls{capability tag} combined with
%     \gls{capability monotonicity}, and will be preserved whether a
%     capability is held in a \gls{capability register} or \gls{tagged memory},
%     subject to suitable use of \gls{capability-based instructions}
The property that a valid-for-use \gls{capability} can only be
    constructed by deriving it from another valid capability
    using a valid capability operation.
% PS: not totally clear what a ``valid capability operation'' is.  An
% execution of a capability instruction that doesn't raise an exception?
    Provenance is implemented using a \gls{capability tag} combined with
    \gls{capability monotonicity}, 
% PS: the text (both previous version and mine) defines provenance as
% a property of the architecture, not ``the provenance of a
% capability'' as the source capability or derivation chain, so we
% can't say ``will be preserved'' like the text did. 
% Maybe we should define that more explicit notion of provenance, and
% replace this glossary entry with one for ``capability provenance
% preservation'', but I've not for now. 
irrespective of 
%
whether a
    capability is held in a \gls{capability register} or \gls{tagged memory}}
% PS: surely ``capability provenance'' should hold universally, not only
%    ``subject to suitable use of \gls{capability-based instructions''?
}

\newglossaryentry{capability register}
{
  name=capability register,
  description={A capability register is an architectural register able to hold
    a \gls{capability} including its \gls{capability tag}, \gls{address},
    other \glslink{fat pointer}{fat-pointer} metadata such as
    its \gls{capability bounds} and \gls{capability permissions}, and optional
    \gls{capability object type}.
    Capability registers may be held in a \gls{capability register file}, a
    \gls{merged register file}, or be a \gls{special capability register}
    accessed by dedicated instructions.
    A capability register might be a dedicated register intended primarily for
    capability-related operations (e.g., the capability registers described
    in \gls{CHERI-MIPS}), or a general-purpose integer
    register that has been extended with capability metadata (such as the
    \gls{program-counter capability}, or the capability registers described in
    \gls{CHERI-RISC-V} when using a merged register file).
    Capability registers must be used to retain tag bits on capabilities
    transiting through memory, as only \gls{capability-based instructions}
    enforce \gls{capability provenance} and \gls{capability monotonicity}}
}

\newglossaryentry{capability register file}
{
  name=capability register file,
  description={The capability register file is a register file dedicated to
    holding general-purpose \glspl{capability}, in contrast to a \gls{merged
    register file}, in which general-purpose integer registers are extended to
    be able to hold tagged capabilities.
    Some general-purpose capability registers have well-known conventions for
    their use in software, including the \gls{return capability} and the
    \gls{stack capability}}
}

\newglossaryentry{capability tag}
{
  name=capability tag,
  description={A capability tag is a 1-bit integrity tag associated with each
    \gls{capability register}, and also with each capability-sized,
    capability-aligned location in memory.
    If the tag is set, the \gls{capability} is valid and can be
    \glslink{dereference}{dereferenced} via the ISA.
    If the tag is clear, then the capability is invalid and cannot be
    dereferenced via the ISA.
    Tags are preserved 
by ISA
operations that conform to \gls{capability
    provenance} and \gls{capability monotonicity} rules -- for example,
    that any attempted modification of \gls{capability bounds} leads to
    non-increasing bounds,
%was ``writes'', not ``bounds'' - presume just a typo?
 and that in-memory capabilities are written only
    via capability stores, not data stores -- otherwise, tags are cleared}
%
%    Subject to these constraints, tags will be preserved by
%    \gls{capability-based instructions}
}

\newglossaryentry{capability validity}
{
  name=capability validity,
  description={A \gls{capability} is valid if its \gls{capability tag}
    is set, which permits use of the capability subject to its
    \gls{capability bounds}, \gls{capability permissions}, and so on.
    Attempts to \gls{dereference} a capability without a tag set will lead
    to a hardware exception}
}

\newglossaryentry{capability-based instructions}
{
  name=capability-based instructions,
  description={These instructions accept capabilities as operands, allowing
    capabilities to be loaded from and stored memory, manipulated subject to
    \gls{capability provenance} and \gls{capability monotonicity} rules,
    and used for a variety of operations such as loading and storing data and
    capabilities, as branch targets, and to retrieve and manipulate capability
    fields -- subject to \gls{capability permissions}}
}

\longnewglossaryentry{CInvoke}
{
  name=CInvoke
}
{
  The \insnref{CInvoke} instruction is a source of controlled
  non-monotonicity in the \gls{CHERI-MIPS} and \gls{CHERI-RISC-V} ISAs.
\psnote{See Kyndylan's note for capability monotonicity}
  It can directly enter any userspace domain described by a pair
  of sealed capabilities with the \emph{Permit\_CInvoke} permission set.
  In particular, it can
  safely enter userspace domain-transition code
  described by the sealed \gls{code capability} while also unsealing
  the sealed \gls{data capability}.
  The sealed operand \glspl{capability register}
  are checked for suitable properties and correspondence, and the userspace
  domain-transition routine can store any return information, perform further error
  checking, and so on.
}

\newglossaryentry{CHERI Concentrate}
{
  name=CHERI Concentrate,
  description={CHERI Concentrate is a specific \gls{compressed capability}
    format that represents a 64-bit \gls{address} with full precision, and
    \gls{capability bounds} relative to that address with reduced precision.
    Bounds have a floating-point representation, requiring that as the size of
    a bounded object increases, greater alignment of its \gls{capability base}
    and \gls{capability length} are required.
    CHERI Concentrate is the successor compression format to \gls{CHERI-128}}
}

\newglossaryentry{CHERI-128}
{
  name=CHERI-128,
  description={CHERI-128 is a specific \gls{compressed capability} format that
    represents a 64-bit \gls{address} with full precision, and
    \gls{capability bounds} relative to that address with reduced precision.
    Bounds have a floating-point representation, requiring that as the size of
    a bounded object increases, greater alignment of its \gls{capability base}
    and \gls{capability length} are required.
    CHERI-128 has been replaced with \gls{CHERI Concentrate}}
}

\newglossaryentry{CHERI-MIPS}
{
  name=CHERI-MIPS,
  description={An application of the CHERI protection model to the 64-bit MIPS
    ISA}
}

\newglossaryentry{CHERI-RISC-V}
{
  name=CHERI-RISC-V,
  description={An application of the CHERI protection model to the RISC-V ISA}
}

\newglossaryentry{CHERI-x86-64}
{
  name=CHERI-x86-64,
  description={An application of the CHERI protection model to the x86-64 ISA}
}

\newglossaryentry{code capability}
{
  name=code capability,
  plural=code capabilities,
  description={A \gls{capability} whose \gls{capability permissions} have been
    configured to permit instruction fetch (i.e., execute) rights; typically,
    write permission will not be granted via an executable capability, in
    contrast to a \gls{data capability}.
    Code capabilities are used to implement \gls{control-flow robustness} by
    constraining the available branch and jump targets}
}

\newglossaryentry{compressed capability}
{
  name=compressed capability,
  plural=compressed capabilities,
  description={A \gls{capability} whose \gls{capability bounds} are
    compressed with respect to its \gls{address}, allowing its
    in-memory footprint to be reduced -- e.g., to 128 bits, rather than the
    roughly
    architectural 256 bits visible to the instruction set when a capability
    is loaded into a register file.
    Certain architecturally valid \glslink{out of bounds}{out-of-bounds}
    addresses may not be \glslink{representable
    capability}{representable} with capability compression; operations leading
     to \glslink{unrepresentable capability}{unrepresentable capabilities}
    will clear the \gls{capability tag} or throw an exception in order to
    ensure continuing \gls{capability monotonicity}.
    \gls{CHERI-128} and \gls{CHERI Concentrate} are specific compressed
    capability models that select particular points in the tradeoff space
    around in-memory capability size, bounds alignment requirements, and
    representability}
}

\newglossaryentry{control-flow robustness}
{
  name=control-flow robustness,
  description={The use of \glspl{code capability} to constrain the set of
    available branch and jump targets for executing code, such that the
    potential for attacker manipulation of the \gls{program-counter
    capability} to simulate injection of arbitrary code is severely
    constrained; a form of \gls{vulnerability mitigation} implemented via
    the \gls{principle of least privilege}}
}

\newglossaryentry{data capability}
{
  name=data capability,
  plural=data capabilities,
  description={A \gls{capability} whose \gls{capability permissions} have been
    configured to permit data load and store, but not instruction fetch (i.e.,
    execute) rights; in contrast to a \gls{code capability}}
}

\newglossaryentry{default data capability}
{
  name=default data capability (\DDC{}),
  description={A \gls{special capability register} constraining
    \glslink{legacy instructions}{legacy} non-\gls{capability-based
    instructions} that load and store data without awareness of the capability
    model.
    Any attempts to load and store will be relocated relative to the default
    data capability's \gls{capability base} and \gls{capability offset}, and
    controlled by its \gls{capability bounds} and \gls{capability
    permissions}.
    Use of the default data capability  violates the \gls{principle of
    intentional use}, but permits compatibility with legacy software.
    A suitably configured default data capability will prevent the use of
    non-capability-based load and store instructions}
}

\newglossaryentry{dereference}
{
  name=dereference,
  description={Dereferencing a \gls{address} means that it is the
    target address for a load, store, or instruction fetch.
    A \gls{capability} may be dereferenced only subject to it being valid
    -- i.e., that its \gls{capability tag} is present --  and is also subject
    to appropriate checks of its \gls{capability bounds}, \gls{capability permissions}, and
    so on.
    Dereference may occur as a result of explicit use of a capability via
    \gls{capability-based instructions}, or implicitly as a result of the
    \gls{program-counter capability} or \gls{default data capability}}
}

\newglossaryentry{exception program-counter capability}
{
  name=exception program-counter capability (\EPCC{}),
  description={A \gls{special capability register} into which the running
    \gls{program-counter capability} will be moved into on an exception, and
    whose value will be moved back into the program-counter capability on
    exception return}
}

\newglossaryentry{fat pointer}
{
  name=fat pointer,
  description={A \gls{pointer} (\gls{address}) that has been extended
    with additional metadata such as \gls{capability bounds} and
    \gls{capability permissions}.
    In conventional fat-pointer designs, fat pointers to not have a notion of
    sealing (i.g., as in \glspl{sealed capability} and \glspl{unsealed
    capability}), nor rules implementing \gls{capability provenance} and
    \gls{capability monotonicity}}
}

\newglossaryentry{fine-grained memory protection}
{
  name=fine-grained memory protection,
  description={The granular description of available code and data in which
    \gls{capability bounds} and \gls{capability permissions} are made as
    small as possible, in order to limit the potential effects of software
    bugs and vulnerabilities.
    This approach applies both to \glspl{code capability} and \glspl{data
    capability}, offering effective \gls{vulnerability mitigation} via
    techniques such as \gls{control-flow robustness}, as well as supporting
    higher-level mitigation techniques such as \gls{software
    compartmentalization}.
    Fine-grained memory protection will typically be driven by the goal of
    implementing the \gls{principle of least privilege}}
}

\newglossaryentry{hybrid capability model}
{
  name=hybrid capability model,
  description={A \gls{capability} model in which not all interfaces to use or
    manipulate capabilities conform to the \gls{principle of intentional
    use}, such that legacy software is able to execute around, or within,
    capability-constrained environments, as well as other features required
    to improve compatibility with conventional software designs permitting
    easier incremental adoption of a capability-system model.
    In CHERI, composition of the capability-system model with the conventional
    Memory Management Unit (MMU), the support for \gls{legacy instructions}
    via the \gls{program-counter capability} and \gls{default data
    capability}, and strong compatibility with the C-language \gls{pointer}
    model, all constitute hybrid aspects of its design, in comparison to a
    more pure capability-system model that might elide those behaviors at a
    cost to compatibility and adoptability}
}

\newglossaryentry{principle of intentional use}
{
  name=principle of intentional use,
  description={A design principle in capability systems in which rights are
    always explicitly, rather than implicitly exercised.
    This arises in the CHERI instruction set through explicit \gls{capability}
    operands to \gls{capability-based instructions}, which contributes to the
    effectiveness of \gls{fine-grained memory protection} and
    \gls{control-flow robustness}.
    When applied, the principle limits not just the rights available in the
    presence of a software vulnerability, but the extent to which software can
    be manipulated into using rights in an unintended (and exploitable)
    manner}
}

\newglossaryentry{invoked data capability}
{
  name=invoked data capability (\IDC{}),
  plural=invoked data capabilities,
  description={A capability register reserved by convention to hold the
    unsealed \gls{data capability} on the callee side of \gls{CInvoke}.
    Typically, for the caller side, this will point at a frame on the caller
    stack sufficient to safely restore any caller state.
    On the callee side, the invoked data capability will be a data capability
    describing the object's internal state}
}

\newglossaryentry{kernel code capability}
{
  name=kernel code capability (\KCC{}),
  description={A \gls{special capability register} reserved to hold a
    privileged \gls{code capability} for use by the kernel during exception
    handling.
    This value will be installed in the \gls{program-counter capability} on
    exception entry, with the previous value of the program-counter
    capability stored in the \gls{exception program-counter capability}}
}

\newglossaryentry{kernel data capability}
{
  name=kernel data capability (\KDC{}),
  description={A \gls{special capability register} reserved to hold a
    privileged \gls{data capability} for use by the kernel during exception
    handling.
    Typically, this will refer either to the data segment for a microkernel
    intended to field exceptions, or for the full kernel.
    Kernels compiled to primarily use \gls{legacy instructions} might install
    this in the \gls{default data capability} for the duration of kernel
    execution.
    Use of this register is controlled by \gls{capability permissions} on
    the currently executing \gls{program-counter capability}}
}

\newglossaryentry{kernel reserved capabilities}
{
  name=kernel reserved capabilities,
  description={These \glspl{capability}, modeled on the MIPS kernel reserved
    registers, are set aside for use by a \gls{CHERI-MIPS} operating-system
    kernel in
    exception handling -- in particular, in allowing userspace registers to
    be saved so that the kernel context can be installed.
    As with the MIPS registers, the userspace ABI is not able to use
    capability registers set aside for kernel use; unlike the MIPS registers,
    the kernel reserved capabilities are available for use in the ISA only
    with a suitably authorized \gls{program-counter capability} installed.
    Due to a different exception-handling model in \gls{CHERI-RISC-V}, that
    ISA does not have kernel reserved capabilities}
}

\newglossaryentry{legacy instructions}
{
  name=legacy instructions,
  description={Legacy instructions are those that accept integer addresses,
    rather than capabilities, as their operands, requiring use of the
    \gls{default data capability} for loads and stores, or that explicitly set
    the program counter to a address, rather than doing setting the
    \gls{program-counter capability}.
    These instructions allow legacy binaries (those compiled without CHERI
    awareness) to execute, but only without the benefits of
    \gls{fine-grained memory protection}, granular \gls{control-flow
    robustness}, or more efficient \gls{software compartmentalization}.
    While still constrained, these instructions do not conform to the
    \gls{principle of intentional use}}
}

\newglossaryentry{merged register file}
{
  name=merged register file,
  description={A single general-purpose register file able to hold both
    integer and tagged \gls{capability} values.
    In \gls{CHERI-MIPS}, a dedicated \gls{capability register file} is used,
    separate from the general-purpose integer register file.
    In \gls{CHERI-RISC-V} and \gls{Morello}, a merged register file is supported, reducing the
    amount of control logic required for a separate register file}
}

\newglossaryentry{Morello}
{
  name=Morello,
  description={An application of the CHERI protection model to the ARMv8-A architecture}
}

\newglossaryentry{out of bounds}
{
  name=out of bounds,
  description={When a \gls{capability}'s \gls{capability offset} falls outside
    of its \gls{capability bounds}, it is out of bounds, and cannot be
    \glslink{dereference}{dereferenced}.
    Even if a capability's offset is in bounds, the width of a data access may
    cause a load, store, or instruction fetch to fall out of bounds, or the
    further offset introduced via a register index or immediate operand to an
    instruction.
%    With 256-bit capabilities, all out-of-bounds pointers are
%    \glspl{representable capability}.
    With \glspl{compressed capability}, if an instruction shifts the offset
    too far out of bounds, this may result in an \gls{unrepresentable
    capability}, leading to the \gls{capability tag} being cleared, or an
    exception being thrown}
}

\newglossaryentry{physical address}
{
  name=physical address,
  plural=physical addresses,
  description={An \gls{address} that is passed directly to the memory
    hierarchy without \glslink{virtual address}{virtual-address} translation.
    In \gls{CHERI-MIPS}, \glspl{capability} contain only virtual addresses.
    In \gls{CHERI-RISC-V}, \glspl{capability} addresses may be interpreted as
    physical addresses in Machine Mode}
}

\newglossaryentry{pointer}
{
  name=pointer,
  description={A pointer is a language-level reference to a memory object.
    In conventional ISAs, a pointer is typically represented as an
    \gls{address}.
    In CHERI, pointers can be represented either as an address
    indirected via the \gls{default data capability} or \gls{program-counter
    capability}, or as a \gls{capability}.
    In the latter cases, its integrity and \gls{capability provenance} are
    protected by the \gls{capability tag}, and its use is limited by
    \gls{capability bounds} and \gls{capability permissions}.
    \Gls{capability-based instructions} preserve the tag as required across
    both \glspl{capability register} and \gls{tagged memory}, and also
    enforce \gls{capability monotonicity}: legitimate operations on the
    pointer cannot broaden the set of rights described by the capability}
}

\newglossaryentry{principle of least privilege}
{
  name=principle of least privilege,
  description={A principle of software design in which the set of rights
    available to running code is minimized to only those required for it to
    function, often with the aim of \gls{vulnerability mitigation}.
    In CHERI, this concept applies via fine-grained memory protection for
    both data and code, and also higher-level \gls{software
    compartmentalization}}
}

\newglossaryentry{program-counter capability}
{
  name=program-counter capability (\PCC{}),
  description={A \gls{special capability register} that extends the existing
    program counter to include
    \gls{capability} metadata such as a \gls{capability tag}, \gls{capability
    bounds}, and \gls{capability permissions}.
    The program-counter capability ensures that instruction fetch occurs only
    subject to capability protections.
    When an exception fires, the value of the program-counter capability will
    be moved to the \gls{exception program-counter capability}, and the value
    of the \gls{kernel data capability} moved into the program-counter
    capability.
    On exception return, the value of the exception program-counter capability
    will be moved into the program-counter capability}
}

\newglossaryentry{representable capability}
{
  name=representable capability,
  plural=representable capabilities,
  description={A \gls{compressed capability} whose \gls{capability offset}
    is representable with respect to its \gls{capability bounds}; this
    does not imply that the offset is ``within bounds'', but does require
    that it be within some broader window around the bounds}
}

\newglossaryentry{reserved capability object type}
{
  name=reserved capability object type,
  plural=reserved capability object types,
  description={Certain \glspl{capability object type} are not available for software use and instead have hardware-defined semantics.
  On \gls{CHERI-MIPS} and \gls{CHERI-RISC-V}, all negative \glspl{capability object type} are
  reserved: \glspl{unsealed capability} use the value $2^{64}-1$ and \glspl{sealed entry capability}
  have an object type of $2^{64}-2$.
  The remaining \glspl{capability object type} are used for \glspl{sealed capability with an object type}}
}

\newglossaryentry{return capability}
{
  name=return capability,
  plural=return capabilities,
  description={A \gls{capability} designated as the destination for the
    return address when using a capability jump-and-link instruction.
    A degree of \gls{control-flow robustness} is provided due to
    \gls{capability bounds}, \gls{capability permissions}, and the
    \gls{capability tag} on the resulting capability, which limits sites that
    may be jumped back to using the return capability}
}

\newglossaryentry{sealed capability}
{
  name=sealed capability,
  plural=sealed capabilities,
  description={A sealed \gls{capability} is one whose \gls{capability object type}
    is not equal to the unsealed object type ($2^{64}-1$ for \gls{CHERI-MIPS} and \gls{CHERI-RISC-V}).
    A sealed capability's \gls{address}, \gls{capability bounds},
    \gls{capability permissions}, and other fields are immutable -- i.e.,
    cannot be modified using \gls{capability-based instructions}.
    A sealed capability cannot be directly \glslink{dereference}{dereferenced}
    using the instruction set, and must be unsealed before it can be used.
    This can be used to implement non-monotonic domain transition, as a
    sealed capability may carry rights not otherwise present in the
    \gls{capability register file}.
    Two types exist: \glspl{sealed capability with an object type} and
    \glspl{sealed entry capability}.
    They have different properties catering to different use cases}
}

\newglossaryentry{sealed capability with an object type}
{
  name=sealed capability with an object type,
  plural=sealed capabilities with object types,
  description={A \gls{sealed capability} whose \gls{capability object type}
    is not one of the \glspl{reserved capability object type}.
    These sealed capability have a \gls{capability object type} derived
    from their \glspl{sealing capability}'s \gls{address}.
    CHERI's sealing feature allows capabilities to be used to describe
    software-defined objects, permitting implementation of encapsulation.
    Unsealing can be performed using the \gls{CInvoke} instruction, or
    using the \insnref{CUnseal} instruction combined with a suitable
    \gls{sealing capability}.
    Sealed capabilities with object types provide the necessary architectural
    encapsulation support to efficiently implement fine-grained
    compartmentalization using an object-oriented model}
}

\newglossaryentry{sealed entry capability}
{
  name=sealed entry capability,
  plural=sealed entry capabilities,
  description={A sealed entry \gls{capability} (also known as
    \gls{sentry capability}) is a \gls{sealed capability}
    whose \gls{capability object type} is set to the sentry \gls{reserved capability object type} ($2^{64}-2$ for \gls{CHERI-MIPS} and \gls{CHERI-RISC-V}).
    Sealed entry capabilities are commonly referred to as \glspl{sentry
    capability}.
    Sealed entry capabilities are do not support linking sealed code and
    data capabilities, unlike \glspl{sealed capability with an object type}.
    A sealed entry capability is unsealed by jumping to it using a regular
    capability jump instruction}
}

\newglossaryentry{sealing capability}
{
  name=sealing capability,
  plural=sealing capabilities,
  description={A sealing capability is one with the \cappermSeal 
    permission, allowing it to be used to create \glspl{sealed capability}
    using a \gls{capability object type} set to the sealing capability's
    \gls{address}, and subject to its bounds}
}

\newglossaryentry{sentry capability}
{
  name=sentry capability,
  plural=sentry capabilities,
  description={Sentry capability is a convenient shorthand for a
    \gls{sealed entry capability}}
}

\newglossaryentry{software compartmentalization}
{
  name=software compartmentalization,
  description={The configuration of \glspl{code capability} and \glspl{data
    capability} available via the \gls{capability register file} or
    \gls{merged register file}, accessible \glspl{special capability
    register}, and \gls{tagged memory} such that software components can be
    isolated from one another, enabling \gls{vulnerability mitigation} via the
    application of the \gls{principle of least privilege} at the application
    layer.
    One approach to implementing software compartmentalization on CHERI is to
    use \gls{CInvoke} to jump into sealed code
    and data capabilities describing a trusted intermediary and destination
    protection domain}
}

\newglossaryentry{stack capability}
{
  name=stack capability,
  plural=stack capabilities,
  description={A \gls{capability} referring to the current stack, whose
    \gls{capability bounds} are suitably configured to allow access only to
    the remaining stack available to allocate at a given point in execution}
}

\newglossaryentry{special capability register}
{
  name=special capability register,
  description={Special capability registers have special architectural
    meanings, and include the \gls{program-counter capability}, the
    \gls{default data capability}, the \gls{exception program-counter
    capability}, the \gls{kernel code capability}, and the \gls{kernel data
    capability}.
    Not all registers are accessible at all times; for example, some may be
    available only in certain rings, or when \PCC{} has the
    Access\_System\_Registers permission set}
}

\newglossaryentry{tagged memory}
{
  name=tagged memory,
  description={Tagged memory associates a 1-bit \gls{capability tag} with
    each \gls{capability}-aligned, capability-sized word in memory.
    \Gls{capability-based instructions} that load and store capabilities
    maintain the tag as the capability transits between memory and the
    \gls{capability register file}, tracking \gls{capability provenance}.
    When data stores (i.e., stores of non-capabilities), the tag on the
    memory location will be atomically cleared, ensuring the integrity of
    in-memory capabilities}
}

\newglossaryentry{trusted computing base}
{
  name=Trusted Computing Base (TCB),
  description={The subset of hardware and software that is critical to the
    security of a system;
    in secure system designs, there is often a goal to minimize the size of
    the TCB in order to minimize the opportunity for exploitable software
    vulnerabilities}
}

\newglossaryentry{trusted stack}
{
  name=trusted stack,
  description={Some software-defined object-capability models offer strong
    call-return semantics -- i.e., that if a return is issued by an invoked
    object, or an uncaught exception is generated, then the appropriate caller
    will be returned to -- exactly once.
    This can be implemented via a trusted stack, maintained by the software
    \gls{trusted computing base} via one or more handlers invoked by \gls{CInvoke}.
    A trusted stack for an object-oriented model will likely maintain at least
    the caller's \gls{program-counter capability} and \gls{invoked data
    capability} to be restored on return}
}

\newglossaryentry{unrepresentable capability}
{
  name=unrepresentable capability,
  plural=unrepresentable capabilities,
  description={A \gls{compressed capability} whose \gls{capability offset} is
    sufficiently outside of its \gls{capability bounds} that the combined
    \gls{pointer} value and bounds cannot be represented in the compressed format;
    constructing an unrepresentable capability will lead to the tag being
    cleared (and information loss) or an exception, rather than a violation
    of \gls{capability provenance} or \gls{capability monotonicity}}
}

\newglossaryentry{unsealed capability}
{
  name=unsealed capability,
  plural=unsealed capabilities,
  description={An unsealed \gls{capability} is one whose \gls{capability object type}
    is the unsealed object type ($2^{64}-1$ for \gls{CHERI-MIPS} and \gls{CHERI-RISC-V}).
    Its remaining capability fields are mutable, subject to \gls{capability
    provenance} and \gls{capability monotonicity} rules.
    These capabilities have hardware-defined behaviors -- i.e., subject to
    \gls{capability bounds}, \gls{capability permissions}, and so on,
    can be \glslink{dereference}{dereferenced}}
}

\newglossaryentry{virtual address}
{
  name=virtual address,
  plural=virtual addresses,
  description={An integer \gls{address} translated by the Memory Management
    Unit (MMU) into a \gls{physical address} for the purposes of load, store,
    and instruction fetch.
    \Glspl{capability} embed an address, represented in the instruction
    set as the sum of the \gls{capability base} and \gls{capability offset},
    as well as \gls{capability bounds} relative to the address.
    The integer addresses passed to \glslink{legacy instructions}{legacy load
    and store instructions} that would previously have been interpreted as
    virtual addresses are, with CHERI, transformed (and checked) using the
    \gls{default data capability}.
    Similarly, the integer addresses passed to legacy branch and jump
    instructions are transformed (and checked) using the \gls{program-counter
    capability}.
    This in effect introduces a further relocation of legacy addresses prior
    to virtual address translation}
}

\newglossaryentry{vulnerability mitigation}
{
  name=vulnerability mitigation,
  description={A set of techniques limiting the effectiveness of the attacker
    to exploit a software vulnerability, typically achieved through use of
    the \gls{principle of least privilege} to constrain injection of
    arbitrary code, control of the \gls{program-counter capability} via
    \gls{control-flow robustness} using \glspl{code capability}, minimization of
    data rights granted via available \glspl{data capability}, and higher-level
    \gls{software compartmentalization}}
}



%% bibliography setup:
% UK date format in bibliography:
\usepackage[british]{babel}
\usepackage{csquotes} % recommended for biblatex
% list up to 99 names instead of the default 3 and set
% giveninits=true to match the abbrv bibtex style.
\usepackage[backend=bibtex,bibencoding=utf8,style=numeric,sortcites,maxnames=99,giveninits=true]{biblatex}
\addbibresource{cheri.bib}
% Note: \citetitle formats the title differently depending on the type of entry,
% whereas this macro always uses \textit{}
\newcommand*{\citetitleit}[1]{\textit{\citefield{#1}{title}}}


% Skip unncessary bibtex fields in the bibliography
\AtEveryBibitem{%
\clearfield{issn}%
\clearfield{urldate}%
\clearfield{urlyear}%
\clearfield{review}%
\clearfield{series}%
\clearfield{note}%
\clearfield{address}%
% avoid printing both isbn and DOI
\iffieldundef{doi}{}{\clearfield{isbn}}%
% we don't want 15 JJ Thomson Avenue, Cambridge for every techreport
% Note: location is a list not a field so we need \clearlist
\clearlist{location}%
}
}\makeatother % end of \@ifclassloaded{standalone}
