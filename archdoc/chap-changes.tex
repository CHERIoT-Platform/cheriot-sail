\newcommand{\ghissue}[1]{\href{https://github.com/microsoft/cheriot-sail/issues/#1}{Issue #1}}
\newcommand{\ghpr}[1]{\href{https://github.com/microsoft/cheriot-sail/pull/#1}{PR #1}}
\chapter{Version history}
\label{chap:changes}

\begin{description}
\item[0.5] The version released as technical report MSR-TR-2023-6: \emph{CHERIoT: Rethinking security for low-cost embedded systems}, February 2023\footnote{\url{https://aka.ms/cheriot-tech-report}}.
\item[0.6] The current, under-development version of the ISA. The following changes have been made since the previous released version:
    \begin{description}
    \item[\ghissue{20}, \ghpr{26}] Capability stores now clear the tag of the stored value instead of raising an exception in case of a store-local violation
    (i.e. an attempt to store a non-global capability via a capability without the store-local permission).
    Tag clearing is preferable for software because it removes the possibility of a trap when copying untrusted inputs.
    It is also likely easier to implement in hardware.
    The capability exception code that was previously used for this (0x16) is now reserved.
    \item[] The relocations for \insnref{auicgp} and \insnref{auipcc} are unified and the CHERIoT-specific relocations are now named with CHERIOT, rather than CHERI, as the prefix.
    \end{description}
\end{description}
