\chapter{Proposed compressed instruction encoding changes}
\label{chap:c-changes}

\cref{sec:c-extension} introduces changes to map certain compressed instructions to their CHERI counterparts, without changing the encoding of compressed instructions themselves.
This brings minimum changes to the instruction decoder of an existing RISC-V CPU.
However, further code size reduction can be achieved if we take advantage of the extra register index bits freed up by RV32E, or if we drastically redesign certain encodings in C extension entirely.

\section{Compressed \asm{CMove} and \asm{CIncAddr}}
The compiler generates \asm{CMove} and \asm{mv} to copy registers for capabilities and integers respectively.
While \asm{mv} has a compressed \asm{c.mv} counterpart, \asm{CMove} does not.
Move instructions are often used to shuffle registers among argument, temporary and callee-saved registers for function calls, taking a significant proportion of the code size.

We take advantage of the freed bits from RV32E and use one bit to differentiate between \asm{c.CMove} and \asm{c.mv}.
Initial code size investigations show a reduction of 5\%, which is significant.
An alternative for \asm{c.CMove} is to conflate \asm{c.CMove} with \asm{c.mv}, which requires no extra bits from RV32E.
However, implications of this conflation on ABI, architectural state and compiler have not been investigated.

Similarly, the extra bit in \asm{c.addi} can encode \asm{c.CIncAddr}.
The benefit of a \asm{c.CIncAddr} is less significant, at around 1-2\%.

\section{Three-operand compressed instructions}
Currently, RISC-V compressed instructions take at most two operands, with certain instructions having full 5-bit register indices to address all 32 registers.
Initial prototyping suggests that sacrificing the ability to address all registers and reducing immediate ranges to increase the number of operands gives further code size reduction.
Increasing the number of operands increases the chance of pattern-matching uncompressed instructions with compressed ones, and the code size reduction outweighs the number of instructions that can no longer be compressed with a smaller immediate or register index.
We currently see another 1-2\% reduction on top of compressed \asm{c.CMove} and \asm{c.CIncAddr}.
However, this is a drastic ISA change and needs further investigation for justification.
