{
\setlength{\parindent}{0cm}

\ifcsname @def@riscv@insns@tex\endcsname
  \ea\endinput
\fi
\ea\gdef\csname @def@riscv@insns@tex\endcsname{1}

\ifcsname @def@riscv@insns@macros@tex\endcsname
  \ea\endinput
\fi
\ea\gdef\csname @def@riscv@insns@macros@tex\endcsname{1}

\makeatletter
% class func name tablestr
\newcommand{\@rvcheriencdef}[4]{%
  \ea\ifx\csname @rvcheri@enc@func2name@#1@\the\numexpr(#2)\endcsname\relax%
    \csgdef{@rvcheri@enc@func2name@#1@\the\numexpr(#2)}{#3}%
    \csgdef{@rvcheri@enc@func2tablestr@#1@\the\numexpr(#2)}{#4}%
  \else%
    \def\@rvcheriencdef@duperror##1{%
      \GenericError{[RISC-V (#1)] }{Duplicate encoding}{%
        [RISC-V (#1)] Function code 0x##1 for #3\MessageBreak%
        is already assigned to \csname @rvcheri@enc@func2name@#1@\the\numexpr(#2)\endcsname%
      }{}%
    }%
    \ea\@rvcheriencdef@duperror\ea{\hex{\the\numexpr(#2)}}%
  \fi%
}
\newcommand{\@rvcheriencusetablestr}[2]{\csuse{@rvcheri@enc@func2tablestr@#1@\the\numexpr(#2)}}
\newcommand{\@rvcherimakeencusetablestrcmd}[1]{%
  \ea\newcommand\csname @rvcheriencusetablestr@#1\endcsname[1]{%
    \@rvcheriencusetablestr{#1}{##1}%
  }%
}
\@rvcherimakeencusetablestrcmd{top}
\@rvcherimakeencusetablestrcmd{srcsrcdest}
\@rvcherimakeencusetablestrcmd{srcsrc}
\@rvcherimakeencusetablestrcmd{src}
\@rvcherimakeencusetablestrcmd{srcdest}
\@rvcherimakeencusetablestrcmd{dest}
\@rvcherimakeencusetablestrcmd{expload}
\@rvcherimakeencusetablestrcmd{expstore}

\let\@rvcherisubclass@tablesuffix\@empty
\define@key{@rvcherisubclass}{tablesuffix}{\def\@rvcherisubclass@tablesuffix{#1}}
\newcommand{\rvcherisubclass}[4][]{{%
  \setkeys{@rvcherisubclass}{#1}%
  \def\@rvcheriencdef@partial##1{\@rvcheriencdef{#2}{"#3}{#4}{#4##1}}%
  \ea\@rvcheriencdef@partial\ea{\@rvcherisubclass@tablesuffix}%
}}

\newbool{@rvcheri@header}
\newcommand{\rvcheriheader}{\booltrue{@rvcheri@header}}
\newcommand{\@rvcheri@ifheader}[1]{%
  \ifbool{@rvcheri@header}{%
    \global\boolfalse{@rvcheri@header}%
    #1%
  }{}%
}
\newcommand{\@rvcherisrcsrcdest}[4]{
  \begin{bytefield}{32}
    \@rvcheri@ifheader{%
      \bitheader[endianness=big]{0,6,7,11,12,14,15,19,20,24,25,31}\\
    }%
    \bitbox{7}{#1}
    \bitbox{5}{#4}
    \bitbox{5}{#3}
    \bitbox{3}{0x0}
    \bitbox{5}{#2}
    \bitbox{7}{0x5b}
  \end{bytefield}%
}
\newcommand{\@rvcherisrcsrcdestimm}[4]{
  \begin{bytefield}{32}
    \@rvcheri@ifheader{%
      \bitheader[endianness=big]{0,6,7,11,12,14,15,19,20,31}\\
    }%
    \bitbox{12}{#4[11:0]}
    \bitbox{5}{#3}
    \bitbox{3}{#1}
    \bitbox{5}{#2}
    \bitbox{7}{0x5b}
  \end{bytefield}%
}
\newcommand{\@rvcherisrcsrcdestclear}[4]{
  \begin{bytefield}{32}
    \@rvcheri@ifheader{%
      \bitheader[endianness=big]{0,6,7,11,12,14,15,17,18,19,20,24,25,31}\\
    }%
    \bitbox{7}{#1}
    \bitbox{5}{#4}
    \bitbox{2}{#3}
    \bitbox{3}{#2$_{[7:5]}$}
    \bitbox{3}{0x0}
    \bitbox{5}{#2$_{[4:0]}$}
    \bitbox{7}{0x5b}
  \end{bytefield}%
}

\newcommand{\@rvcherisrcsrc}[3]{\@rvcherisrcsrcdest{0x7e}{#1}{#2}{#3}}
\newcommand{\@rvcherisrc}[2]{\@rvcherisrcsrc{0x1f}{#2}{#1}}
\newcommand{\@rvcherisrcdest}[3]{\@rvcherisrcsrcdest{0x7f}{#2}{#3}{#1}}
\newcommand{\@rvcheridest}[2]{\@rvcherisrcdest{0x1f}{#2}{#1}}
\newcommand{\@rvcheriscr}[4]{\@rvcherisrcsrcdest{#1}{#2}{#4}{#3}}
\newcommand{\@rvchericlear}[1]{\@rvcherisrcsrcdestclear{0x7f}{m}{q}{#1}}
\newcommand{\@rvcheristorever}[3]{\@rvcherisrcsrcdest{0x7e}{#1}{#3}{#2}}
\newcommand{\@rvcheriexpload}[3]{\@rvcherisrcsrcdest{0x7d}{#2}{#3}{#1}}
\newcommand{\@rvcheriexpstore}[3]{\@rvcherisrcsrcdest{0x7c}{#1}{#3}{#2}}
\newcommand{\@rvcheriexpstorecond}[4]{\@rvcherisrcsrcdest{0x7c}{#1}{#4}{#2/#3}}

\newcommand{\@rvcheriasmdef}[2]{%
  \ea\ifx\csname @rvcheri@asm@#1\endcsname\relax%
    \csgdef{@rvcheri@asm@#1}{#2}%
  \else%
    \def\@rvcheriasmdef@duperror##1{%
      \GenericError{[RISC-V] }{Duplicate assembly ID}{%
        [RISC-V] Instruction ID ##1 already has an assembly definition%
      }%
    }%
    \@rvcheriasmdef@duperror{#1}%
  \fi%
}

\newcommand{\@rvcheriasmencdef}[3]{%
  \ifthenelse{\equal{\@rvcheriinsn@noref}{true}}{%
    \def\@rvcheriasmencdef@insnref{rvcheriasminsnnoref}%
  }{%
    \def\@rvcheriasmencdef@insnref{rvcheriasminsnref}%
  }%
  \ifthenelse{\equal{\@rvcheriinsn@name}{}}{%
  }{%
    \ea\def%
      \ea\@rvcheriasmencdef@asm@partial%
      \ea##\ea1%
      \ea{%
        \csname rvcheriasmfmt\ea\ea\ea\endcsname%
          \ea\ea\ea[\ea\@rvcheriinsn@restriction\ea]\ea{%
            \csname\@rvcheriasmencdef@insnref\endcsname{##1} #3}}%
    \ea\ea\ea\def%
      \ea\ea\ea\@rvcheriasmencdef@asm%
      \ea\ea\ea{%
        \ea\@rvcheriasmencdef@asm@partial%
        \ea{\@rvcheriinsn@name}}%
    \ea\ea\ea\@rvcheriasmdef\ea\ea\ea{\ea\@rvcheriinsn@id\ea}\ea{\@rvcheriasmencdef@asm}%
    \ifthenelse{\equal{\@rvcheriinsn@id}{\@rvcheriinsn@shortid}}{%
    }{%
      \ea\ea\ea\@rvcheriasmdef\ea\ea\ea{\ea\@rvcheriinsn@shortid\ea}\ea{\@rvcheriasmencdef@asm}%
    }%
    \ifthenelse{\equal{\@rvcheriinsn@notable}{true}}{%
    }{%
      \def\@rvcheriasmencdef@rvcheriencdef@partial{\@rvcheriencdef{#1}{#2}}%
      \ea\def\ea\@rvcheriasmencdef@hypershortname\ea{%
        \csname\@rvcheriasmencdef@insnref\ea\endcsname\ea*\ea{\@rvcheriinsn@shortname}%
      }%
      \ea\ea\ea\def\ea\ea\ea\@rvcheriasmencdef@tablestr\ea\ea\ea{%
        \ea\@rvcheriasmencdef@hypershortname\@rvcheriinsn@tablesuffix%
      }%
      \ea\ea\ea\@rvcheriasmencdef@rvcheriencdef@partial%
        \ea\ea\ea{%
          \ea\@rvcheriinsn@name%
        \ea}%
        \ea{%
          \ea\unexpanded\ea{\@rvcheriasmencdef@tablestr}%
        }%
    }%
  }%
}

\let\@rvcheriinsn@name\@empty
\def\@rvcheriinsn@noref{false}
\def\@rvcheriinsn@notable{false}
\def\@rvcheriinsn@rawfunc{false}
\let\@rvcheriinsn@tablesuffix\@empty
\define@key{@rvcheriinsn}{name}{\def\@rvcheriinsn@name{#1}}
\define@key{@rvcheriinsn}{noref}[true]{\def\@rvcheriinsn@noref{#1}}
\define@key{@rvcheriinsn}{notable}[true]{\def\@rvcheriinsn@notable{#1}}
\define@key{@rvcheriinsn}{rawfunc}[true]{\def\@rvcheriinsn@rawfunc{#1}}
\define@key{@rvcheriinsn}{restriction}{\def\@rvcheriinsn@restriction{#1}}
\define@key{@rvcheriinsn}{shortname}{\def\@rvcheriinsn@shortname{#1}}
\define@key{@rvcheriinsn}{tablesuffix}{\def\@rvcheriinsn@tablesuffix{#1}}
\def\@rvcheriinsnsetkeys#1{%
  \setkeys{@rvcheriinsn}{#1}%
  \ea\ifx\csname @rvcheriinsn@shortname\endcsname\relax%
    \let\@rvcheriinsn@shortname\@rvcheriinsn@name%
  \fi%
  \ea\ifx\csname @rvcheriinsn@name\endcsname\relax%
  \else%
    \ea\def\ea\@rvcheriinsn@id\ea{\@rvcheriinsn@name}%
    \ea\def\ea\@rvcheriinsn@shortid\ea{\@rvcheriinsn@shortname}%
    \ea\ifx\csname @rvcheriinsn@restriction\endcsname\relax%
      \let\@rvcheriinsn@restriction\@empty%
    \else%
      \ea\ea\ea\def\ea\ea\ea\@rvcheriinsn@id\ea\ea\ea{\ea\@rvcheriinsn@id\ea:\@rvcheriinsn@restriction}%
      \ea\ea\ea\def\ea\ea\ea\@rvcheriinsn@shortid\ea\ea\ea{\ea\@rvcheriinsn@shortid\ea:\@rvcheriinsn@restriction}%
    \fi%
  \fi%
}
\def\@rvcheriinsnfmtfunc#1{%
  \ifthenelse{\equal{\@rvcheriinsn@rawfunc}{true}}{%
    #1%
  }{%
    0x\lowercase{#1}%
  }%
}

\newcommand{\@rvcheribitboxdef@single}[2]{%
  \ea\ifx\csname @rvcheri@bitbox@#1\endcsname\relax%
    \csgdef{@rvcheri@bitbox@#1}{#2}%
  \else%
    \def\@rvcheribitboxdef@duperror##1{%
      \GenericError{[RISC-V] }{Duplicate bitbox ID}{%
        [RISC-V] Instruction ID ##1 already has a bitbox definition%
      }%
    }%
    \@rvcheribitboxdef@duperror{#1}%
  \fi%
}

\newcommand{\@rvcheribitboxdef}[1]{%
  \ea\@rvcheribitboxdef@single\ea{\@rvcheriinsn@id}{#1}%
  \ifthenelse{\equal{\@rvcheriinsn@id}{\@rvcheriinsn@shortid}}{%
  }{%
    \ea\@rvcheribitboxdef@single\ea{\@rvcheriinsn@shortid}{#1}%
  }%
}

\def\@rvcherirawbitbox#1{%
  \csname @rvcheri#1\endcsname%
}
\let\rvcherirawbitbox\@rvcherirawbitbox

\newcommand{\rvcherisrcsrcdest}[5][]{{%
  \@rvcheriinsnsetkeys{#1}%
  \@rvcheribitboxdef{\@rvcherirawbitbox{srcsrcdest}{\@rvcheriinsnfmtfunc{#2}}{#3}{#4}{#5}}%
  \@rvcheriasmencdef{srcsrcdest}{"#2}{#3, #4, #5}%
}}
\newcommand{\rvcherisrcsrcdestimm}[5][]{{%
  \@rvcheriinsnsetkeys{#1}%
  \@rvcheribitboxdef{\@rvcherirawbitbox{srcsrcdestimm}{\@rvcheriinsnfmtfunc{#2}}{#3}{#4}{#5}}%
  \@rvcheriasmencdef{top}{"#2}{#3, #4, #5}%
}}
\newcommand{\rvcherisrcsrc}[4][]{{%
  \@rvcheriinsnsetkeys{#1}%
  \@rvcheribitboxdef{\@rvcherirawbitbox{srcsrc}{\@rvcheriinsnfmtfunc{#2}}{#3}{#4}}%
  \@rvcheriasmencdef{srcsrc}{"#2}{#3, #4}%
}}
\newcommand{\rvcherisrc}[3][]{{%
  \@rvcheriinsnsetkeys{#1}%
  \@rvcheribitboxdef{\@rvcherirawbitbox{src}{\@rvcheriinsnfmtfunc{#2}}{#3}}%
  \@rvcheriasmencdef{src}{"#2}{#3}%
}}
\newcommand{\rvcherisrcdest}[4][]{{%
  \@rvcheriinsnsetkeys{#1}%
  \@rvcheribitboxdef{\@rvcherirawbitbox{srcdest}{\@rvcheriinsnfmtfunc{#2}}{#3}{#4}}%
  \@rvcheriasmencdef{srcdest}{"#2}{#3, #4}%
}}
\newcommand{\rvcheridest}[3][]{{%
  \@rvcheriinsnsetkeys{#1}%
  \@rvcheribitboxdef{\@rvcherirawbitbox{dest}{\@rvcheriinsnfmtfunc{#2}}{#3}}%
  \@rvcheriasmencdef{dest}{"#2}{#3}%
}}

\newcommand{\rvcheriscr}[5][]{{%
  \@rvcheriinsnsetkeys{#1}%
  \@rvcheribitboxdef{\@rvcherirawbitbox{scr}{\@rvcheriinsnfmtfunc{#2}}{#3}{#4}{#5}}%
  \@rvcheriasmencdef{srcsrcdest}{"#2}{#3, #4, #5}%
}}
\newcommand{\rvchericlear}[2][]{{%
  \@rvcheriinsnsetkeys{#1}%
  \@rvcheribitboxdef{\@rvcherirawbitbox{clear}{\@rvcheriinsnfmtfunc{#2}}}%
  \@rvcheriasmencdef{srcdest}{"#2}{q(uarter), m(ask)}%
}}

\newcommand{\rvcheristorever}[4][]{{%
  \@rvcheriinsnsetkeys{#1}%
  \@rvcheribitboxdef{\@rvcherirawbitbox{storever}{\@rvcheriinsnfmtfunc{#2}}{#3}{#4}}%
  \@rvcheriasmencdef{srcsrc}{"#2}{#3, #4}%
}}

\newcommand{\rvcheriexpload}[4][]{{%
  \@rvcheriinsnsetkeys{#1}%
  \@rvcheribitboxdef{\@rvcherirawbitbox{expload}{\@rvcheriinsnfmtfunc{#2}}{#3}{#4}}%
  \@rvcheriasmencdef{expload}{"#2}{#3, #4}%
}}
\newcommand{\rvcheriexploadres}[4][]{{%
  \@rvcheriinsnsetkeys{#1}%
  \@rvcheribitboxdef{\@rvcherirawbitbox{expload}{\@rvcheriinsnfmtfunc{#2}}{#3}{#4}}%
  \@rvcheriasmencdef{expload}{"#2}{#3, #4}%
}}
\newcommand{\rvcheriexpstore}[4][]{{%
  \@rvcheriinsnsetkeys{#1}%
  \@rvcheribitboxdef{\@rvcherirawbitbox{expstore}{\@rvcheriinsnfmtfunc{#2}}{#3}{#4}}%
  \@rvcheriasmencdef{expstore}{"#2}{#3, #4}%
}}
\newcommand{\rvcheriexpstorecond}[5][]{{%
  \@rvcheriinsnsetkeys{#1}%
  \@rvcheribitboxdef{\@rvcherirawbitbox{expstorecond}{\@rvcheriinsnfmtfunc{#2}}{#3}{#4}{#5}}%
  \@rvcheriasmencdef{expstore}{"#2}{#4, #5}%
}}

\newcommand{\rvcheribitbox}[1]{%
  \ea\ifx\csname @rvcheri@bitbox@#1\endcsname\relax%
    \def\rvcheribitbox@unknownerr##1{%
      \GenericError{[RISC-V] }{Unknown bitbox ID}{%
        [RISC-V] Instruction ID ##1 has no known bitbox definition%
      }{}%
    }%
    \rvcheribitbox@unknownerr{#1}%
  \else%
    \csname @rvcheri@bitbox@#1\endcsname%
  \fi%
}

\newcommand{\rvcheriasm}[1]{%
  \ea\ifx\csname @rvcheri@asm@#1\endcsname\relax%
    \def\rvcheriasm@unknownerr##1{%
      \GenericError{[RISC-V] }{Unknown assembly ID}{%
        [RISC-V] Instruction ID ##1 has no known assembly definition%
      }{}%
    }%
    \rvcheriasm@unknownerr{#1}%
  \else%
    \csname @rvcheri@asm@#1\endcsname%
  \fi%
}
\makeatother


\rvcherisubclass{top}{0}{Two Source \& Dest}

\rvcherisubclass{srcsrcdest}{7C}{Stores}
\rvcherisubclass{srcsrcdest}{7D}{Loads}
\rvcherisubclass{srcsrcdest}{7E}{Two Source}
\rvcherisubclass{srcsrcdest}{7F}{Source \& Dest}

\rvcherisubclass{srcsrc}{1F}{One Source}

\rvcherisubclass{srcdest}{1F}{Dest-Only}

\rvcherisrcdest[name=CGetPerm]{0}{rd}{cs1}
\rvcherisrcdest[name=CGetType]{1}{rd}{cs1}
\rvcherisrcdest[name=CGetBase]{2}{rd}{cs1}
\rvcherisrcdest[name=CGetLen]{3}{rd}{cs1}
\rvcherisrcdest[name=CGetTag]{4}{rd}{cs1}
\rvcherisrcdest[name=CGetAddr]{F}{rd}{cs1}
\rvcherisrcdest[name=CGetHigh]{17}{rd}{cs1}
\rvcherisrcdest[name=CGetTop]{18}{rd}{cs1}

\rvcherisrcsrcdest[name=CSeal]{B}{cd}{cs1}{cs2}
\rvcherisrcsrcdest[name=CUnseal]{C}{cd}{cs1}{cs2}
\rvcherisrcsrcdest[name=CAndPerm]{D}{cd}{cs1}{rs2}
\rvcherisrcsrcdest[name=CSetAddr]{10}{cd}{cs1}{rs2}
\rvcherisrcsrcdest[name=CIncAddr]{11}{cd}{cs1}{rs2}
\rvcherisrcsrcdestimm[name=CIncAddrImm]{1}{cd}{cs1}{imm}
\rvcherisrcsrcdest[name=CSetBounds]{8}{cd}{cs1}{rs2}
\rvcherisrcsrcdest[name=CSetBoundsExact]{9}{cd}{cs1}{rs2}
\rvcherisrcsrcdestimm[name=CSetBoundsImm]{2}{cd}{cs1}{uimm}
\rvcherisrcsrcdest[name=CSetHigh]{16}{cd}{cs1}{rs2}
\rvcherisrcdest[name=CClearTag]{B}{cd}{cs1}
% \rvcherisrcsrcdest[name=CRelocate,noref,tablesuffix=\rvcherireservedfootnotemark]{15}{cd}{cs1}{rs2}

\rvcherisrcsrcdest[name=CSub]{14}{rd}{cs1}{cs2}
\rvcherisrcdest[name=CMove]{A}{cd}{cs1}

\rvcherisrcsrcdest[name=CTestSubset]{20}{rd}{cs1}{cs2}
\rvcherisrcsrcdest[name=CSetEqualExact,shortname=CSEQX]{21}{rd}{cs1}{cs2}

\rvcheriscr[name=CSpecialRW]{1}{cd}{scr}{cs1}

\rvcherisrcdest[name=CRoundRepresentableLength,shortname=CRRL]{8}{rd}{rs1}
\rvcherisrcdest[name=CRepresentableAlignmentMask,shortname=CRAM]{9}{rd}{rs1}

% \rvcheriexploadres[name=LR.B.CAP,noref,tablesuffix=\rvcheriatomicfootnotemark]{18}{rd}{cs1}
% \rvcheriexploadres[name=LR.H.CAP,noref,tablesuffix=\rvcheriatomicfootnotemark]{19}{rd}{cs1}
% \rvcheriexploadres[name=LR.W.CAP,noref,tablesuffix=\rvcheriatomicfootnotemark]{1A}{rd}{cs1}
% \rvcheriexploadres[name=LR.C.CAP,restriction=RV32,noref,tablesuffix=\rvcheriatomicfootnotemark,notable]{1B}{cd}{cs1}
% \rvcheriexploadres[name=LR.D.CAP,restriction=RV64/128,noref,tablesuffix=\rvcheriatomicfootnotemark]{1B}{rd}{cs1}
% \rvcheriexploadres[name=LR.C.CAP,restriction=RV64,noref,tablesuffix=\rvcheriatomicfootnotemark,notable]{1C}{cd}{cs1}
% \rvcheriexploadres[name=LR.Q.CAP,restriction=RV128,noref,tablesuffix=\rvcheriatomicfootnotemark]{1C}{rd}{cs1}

% \rvcheriexpstorecond[name=SC.B.CAP,noref,tablesuffix=\rvcheriatomicfootnotemark]{18}{rd}{rs2}{cs1}
% \rvcheriexpstorecond[name=SC.H.CAP,noref,tablesuffix=\rvcheriatomicfootnotemark]{19}{rd}{rs2}{cs1}
% \rvcheriexpstorecond[name=SC.W.CAP,noref,tablesuffix=\rvcheriatomicfootnotemark]{1A}{rd}{rs2}{cs1}
% \rvcheriexpstorecond[name=SC.C.CAP,restriction=RV32,noref,tablesuffix=\rvcheriatomicfootnotemark,notable]{1B}{cd}{cs2}{cs1}
% \rvcheriexpstorecond[name=SC.D.CAP,restriction=RV64/128,noref,tablesuffix=\rvcheriatomicfootnotemark]{1B}{rd}{rs2}{cs1}
% \rvcheriexpstorecond[name=SC.C.CAP,restriction=RV64,noref,tablesuffix=\rvcheriatomicfootnotemark,notable]{1C}{cd}{cs2}{cs1}
% \rvcheriexpstorecond[name=SC.Q.CAP,restriction=RV128,noref,tablesuffix=\rvcheriatomicfootnotemark]{1C}{rd}{rs2}{cs1}

\ifcsname @app@isaquick@table@macros@tex\endcsname
  \ea\endinput
\fi
\ea\gdef\csname @app@isaquick@table@macros@tex\endcsname{1}

\makeatletter
\newcount\@cherienctable@col
\newcount\@cherienctable@cols
\newcount\@cherienctable@colbits
\newcount\@cherienctable@row
\newcount\@cherienctable@rows
\newcount\@cherienctable@rowbits
\newcount\@cherienctable@tmp
\def\@cherienctable@addtoformat#1{\ea\global\ea\def\ea\@cherienctable@format\ea{\@cherienctable@format #1}}
\def\@cherienctable@addtobody#1{\ea\global\ea\def\ea\@cherienctable@body\ea{\@cherienctable@body #1}}
% cols func2str count
\newcommand{\@cherienctable}[3]{%
  \@cherienctable@cols=\numexpr(#1)\relax%
  \@cherienctable@rows=\numexpr(#3+\@cherienctable@cols-1)\relax%
  \divide\@cherienctable@rows\@cherienctable@cols%
  %
  \let\@cherienctable@format\@empty%
  \let\@cherienctable@body\@empty%
  \ifnum\@cherienctable@rows>1%
    \@cherienctable@addtoformat{r|}%
    \@cherienctable@addtobody{ & }%
  \fi%
  %
  \@cherienctable@colbits=1%
  \@cherienctable@tmp=2%
  \loop\ifnum\@cherienctable@tmp<\@cherienctable@cols%
    \advance\@cherienctable@colbits 1%
    \multiply\@cherienctable@tmp 2%
  \repeat%
  %
  \@cherienctable@rowbits=1%
  \@cherienctable@tmp=2%
  \loop\ifnum\@cherienctable@tmp<\@cherienctable@rows%
    \advance\@cherienctable@rowbits 1%
    \multiply\@cherienctable@tmp 2%
  \repeat%
  %
  \@cherienctable@col=0%
  \loop\ifnum\@cherienctable@col<\@cherienctable@cols%
    \@cherienctable@addtoformat{c}%
    \ifnum\@cherienctable@col>0%
      \@cherienctable@addtobody{ & }%
    \fi%
    \edef\@cherienctable@cell{\nbinary{\@cherienctable@colbits}{\the\@cherienctable@col}}%
    \ea\@cherienctable@addtobody\ea{\@cherienctable@cell}%
    \advance\@cherienctable@col 1%
  \repeat%
  \@cherienctable@addtobody{ \\ \hline}%
  %
  \@cherienctable@row=0%
  \loop\ifnum\@cherienctable@row<\@cherienctable@rows%
    \ifnum\@cherienctable@rows>1%
      \edef\@cherienctable@cell{\nbinary{\@cherienctable@rowbits}{\the\@cherienctable@row}}%
      \ea\@cherienctable@addtobody\ea{\@cherienctable@cell & }%
    \fi%
    \@cherienctable@col=0%
    {%
      \loop\ifnum\@cherienctable@col<\@cherienctable@cols%
        \ifnum\@cherienctable@col>0%
          \@cherienctable@addtobody{ & }%
        \fi%
        \edef\@cherienctable@cell{\csname #2\endcsname{\@cherienctable@row*\@cherienctable@cols + \@cherienctable@col}}%
        \ifx\@cherienctable@cell\@empty%
          \@cherienctable@addtobody{-}%
        \else%
          \ea\@cherienctable@addtobody\ea{\@cherienctable@cell}%
        \fi%
        \advance\@cherienctable@col 1%
      \repeat%
    }%
    \@cherienctable@addtobody{ \\}%
    \advance\@cherienctable@row 1%
  \repeat%
  %
  \def\@cherienctable@begintabular{\begin{tabular}}%
  \ea\@cherienctable@begintabular\ea{\@cherienctable@format}%
  \@cherienctable@body%
  \end{tabular}%
}
\makeatother


\makeatletter
\def\rvcherienctablecols{4}
\def\rvcherienctablefontsize{\normalsize}

% class count
\newcommand{\@rvcherimakeenctablecmd}[2]{%
  % [cols]
  \ea\NewDocumentCommand\ea{\csname rvcherienctable#1\endcsname}{o}{%
    \IfValueTF{##1}{%
      \@cherienctable{##1}{@rvcheriencusetablestr@#1}{#2}%
    }{%
      \ea\@cherienctable\ea{\rvcherienctablecols}{@rvcheriencusetablestr@#1}{#2}%
    }%
  }%
}
\@rvcherimakeenctablecmd{top}{8}
\@rvcherimakeenctablecmd{srcsrcdest}{128}
\@rvcherimakeenctablecmd{srcsrc}{32}
\@rvcherimakeenctablecmd{src}{32}
\@rvcherimakeenctablecmd{srcdest}{32}
\@rvcherimakeenctablecmd{dest}{32}
\@rvcherimakeenctablecmd{expload}{32}
\@rvcherimakeenctablecmd{expstore}{32}

\let\rvcheriasminsnref\insnriscvref
\let\rvcheriasminsnnoref\insnnoref
\providecommand{\rvcheriasmfmt}{}
\renewcommand{\rvcheriasmfmt}[2][]{%
  ~\raiseforbf{%
    \textsf{\footnotesize{#2}}%
    \ifthenelse{\equal{#1}{}}{%
    }{%
      ~{\textit{\scriptsize{(#1)}}}%
    }%
  }%
}

\newcommand{\rvcheriisaquick}[1]{%
  \rvcheribitbox{#1}~\rvcheriasm{#1}%
}

\newcommand{\riscvbitboxaq}{\rotateinbitbox{\small aq}}
\newcommand{\riscvbitboxrl}{\rotateinbitbox{\small rl}}
\makeatother


\chapter{Instruction encoding summary}
\label{app:isaquick-riscv}

	\section{Primary new instructions}

		The RISC-V specification reserves 4 major opcodes for extensions: 11 (0xb / 0b0001011), 43 (0x2b / 0b0101011), 91 (0x5b / 0b1011011), and 123 (0x7b / 0b1111011).
		The proposed CHERI encodings use major opcode 0x5b for all capability instructions.

		All register-register operations use the RISC-V R-type or I-type encoding formats.
	\optype{Capability-Inspection}

		\rvcheriheader
		\rvcheriisaquick{CGetPerm}

		\rvcheriisaquick{CGetType}

		\rvcheriisaquick{CGetBase}

		\rvcheriisaquick{CGetLen}

		\rvcheriisaquick{CGetTag}

		\rvcheriisaquick{CGetAddr}

		\rvcheriisaquick{CGetHigh}

		\rvcheriisaquick{CGetTop}

	\optype{Capability-Modification}

		\rvcheriheader
		\rvcheriisaquick{CSeal}

		\rvcheriisaquick{CUnseal}

		\rvcheriisaquick{CAndPerm}

		\rvcheriisaquick{CSetAddr}

		\rvcheriisaquick{CIncAddr}

		\rvcheriisaquick{CIncAddrImm}

		\rvcheriisaquick{CSetBounds}

		\rvcheriisaquick{CSetBoundsExact}

		\rvcheriisaquick{CSetBoundsImm}

		\rvcheriisaquick{CSetHigh}

		\rvcheriisaquick{CClearTag}

		\texttt{CSetBoundsExact} may not be required.

	\optype{Pointer-Arithmetic}

		\rvcheriheader

		\jwnote{We do not need CSub, since a standard Sub will return the difference between two capabilities.}

		\jrtcnote{We do need a separate CSub with a split register file though,
		so we define one that should be used even with a merged register file.}

		\rvcheriisaquick{CSub}

		\rvcheriisaquick{CMove}

	\optype{Pointer-Comparison}

		\rvcheriheader

		\rvcheriisaquick{CTestSubset}

		\rvcheriisaquick{CSetEqualExact}

	\optype{Special Capabilty Register Access}
	\rvcheriheader

		\rvcheriisaquick{CSpecialRW}

	\optype{Adjusting to Compressed Capability Precision}
	\rvcheriheader

		\rvcheriisaquick{CRoundRepresentableLength}

		\rvcheriisaquick{CRepresentableAlignmentMask}

	\section{Modifications to existing RISC-V instructions}
	\optype{Control-Flow}

	No special new control flow instructions are added, although RISC-V \texttt{JAL} / \texttt{JALR} become capability instructions \rvcheriasminsnref{CJAL}  / \rvcheriasminsnref{CJALR}.
	The branch instructions also check that \PCC{} permits at least one 2-byte instruction to be loaded from the target address, otherwise they raise a capability bounds exception.

	\optype{Memory-Access}
	\label{quickref:mem}

	\vspace{1.5ex}

The standard RV32 load and store instructions are modified to take a capability
as the base address:\\

		\begin{bytefield}{32}
			\bitheader[endianness=big]{0,6,7,11,12,14,15,19,20,24,25,31}\\
			\bitbox{12}{imm[11:0]}
			\bitbox{5}{cs1}
			\bitbox{3}{op}
			\bitbox{5}{rd}
			\bitbox{7}{0x3}
		\end{bytefield}
		\rvcheriasmfmt{CL[BHW][U] rd, cs1, imm}

		\begin{bytefield}{32}
			\bitbox{7}{imm[11:5]}
			\bitbox{5}{rs2}
			\bitbox{5}{cs1}
			\bitbox{3}{op}
			\bitbox{5}{imm[4:0]}
			\bitbox{7}{0x23}
		\end{bytefield}
		\rvcheriasmfmt{CS[BHW] rs2, cs1, imm}\\

The RV64 instructions \texttt{LD} and \texttt{SD} are reused to behave as load capability (\texttt{LC}) and store capability (\texttt{SC}) respectively:\\

		\begin{bytefield}{32}
			\bitheader[endianness=big]{0,6,7,11,12,14,15,19,20,24,25,31}\\
			\bitbox{12}{imm}
			\bitbox{5}{rs1}
			\bitbox{3}{0x3}
			\bitbox{5}{cd}
			\bitbox{7}{0x3}
		\end{bytefield}
		\rvcheriasmfmt[RV32]{\rvcheriasminsnref{CLC} cd, rs1, imm}

		\begin{bytefield}{32}
			\bitbox{7}{imm[11:5]}
			\bitbox{5}{cs2}
			\bitbox{5}{rs1}
			\bitbox{3}{0x3}
			\bitbox{5}{imm[4:0]}
			\bitbox{7}{0x23}
		\end{bytefield}
		\rvcheriasmfmt[RV32]{\rvcheriasminsnref{CSC} cs2, rs1, imm}\\

% 	\optype{Atomic Memory-Access}

% When using 64-bit capabilities in RV32, the RV64A instructions \texttt{LR.D}, \texttt{SC.D} and \texttt{AMOSWAP.D} are reused to behave as \texttt{LR.C}, \texttt{SC.C} and \texttt{AMOSWAP.C} respectively.\\

% 		\begin{bytefield}{32}
% 			\bitheader[endianness=big]{0,6,7,11,12,14,15,19,20,24,25,26,27,31}\\
% 			\bitbox{5}{0x2}
% 			\bitbox{1}{\riscvbitboxaq}
% 			\bitbox{1}{\riscvbitboxrl}
% 			\bitbox{5}{0x0}
% 			\bitbox{5}{rs1}
% 			\bitbox{3}{0x3}
% 			\bitbox{5}{cd}
% 			\bitbox{7}{0x2f}
% 		\end{bytefield}
% 		\rvcheriasmfmt[RV32]{\rvcheriasminsnnoref{LR.C} cd, rs1}

% 		\begin{bytefield}{32}
% 			\bitbox{5}{0x3}
% 			\bitbox{1}{\riscvbitboxaq}
% 			\bitbox{1}{\riscvbitboxrl}
% 			\bitbox{5}{cs2}
% 			\bitbox{5}{rs1}
% 			\bitbox{3}{0x3}
% 			\bitbox{5}{rd}
% 			\bitbox{7}{0x2f}
% 		\end{bytefield}
% 		\rvcheriasmfmt[RV32]{\rvcheriasminsnnoref{SC.C} rd, cs2, rs1}

% 		\begin{bytefield}{32}
% 			\bitbox{5}{0x1}
% 			\bitbox{1}{\riscvbitboxaq}
% 			\bitbox{1}{\riscvbitboxrl}
% 			\bitbox{5}{cs2}
% 			\bitbox{5}{rs1}
% 			\bitbox{3}{0x3}
% 			\bitbox{5}{cd}
% 			\bitbox{7}{0x2f}
% 		\end{bytefield}
% 		\rvcheriasmfmt[RV32]{\rvcheriasminsnnoref{AMOSWAP.C} cd, cs2, rs1}
%
% We do not provide any of the other AMOs at this point when operating on
% capability values, as they generally make sense only when operating on integer
% values.

% Since capabilities have precise bounds, sub-word atomics cannot be implemented
% using word-sized atomics. To avoid unnecessary complexity compared with a
% non-CHERI RISC-V implementation, we define only \texttt{LR.B}, \texttt{SC.B},
% \texttt{LR.H} and \texttt{SC.H}, without any of the corresponding AMOs. We also
% only require these to be present in capability mode, but implementations may
% choose to always provide them for simplicity.

% 		\begin{bytefield}{32}
% 			\bitheader[endianness=big]{0,6,7,11,12,14,15,19,20,24,25,26,27,31}\\
% 			\bitbox{5}{0x2}
% 			\bitbox{1}{\riscvbitboxaq}
% 			\bitbox{1}{\riscvbitboxrl}
% 			\bitbox{5}{0x0}
% 			\bitbox{5}{rs1}
% 			\bitbox{3}{0x0}
% 			\bitbox{5}{rd}
% 			\bitbox{7}{0x2f}
% 		\end{bytefield}
% 		\rvcheriasmfmt{\rvcheriasminsnnoref{LR.B} rd, rs1}

% 		\begin{bytefield}{32}
% 			\bitbox{5}{0x3}
% 			\bitbox{1}{\riscvbitboxaq}
% 			\bitbox{1}{\riscvbitboxrl}
% 			\bitbox{5}{rs2}
% 			\bitbox{5}{rs1}
% 			\bitbox{3}{0x0}
% 			\bitbox{5}{rd}
% 			\bitbox{7}{0x2f}
% 		\end{bytefield}
% 		\rvcheriasmfmt{\rvcheriasminsnnoref{SC.B} rd, rs2, rs1}

% 		\begin{bytefield}{32}
% 			\bitbox{5}{0x2}
% 			\bitbox{1}{\riscvbitboxaq}
% 			\bitbox{1}{\riscvbitboxrl}
% 			\bitbox{5}{0x0}
% 			\bitbox{5}{rs1}
% 			\bitbox{3}{0x1}
% 			\bitbox{5}{rd}
% 			\bitbox{7}{0x2f}
% 		\end{bytefield}
% 		\rvcheriasmfmt{\rvcheriasminsnnoref{LR.H} rd, rs1}

% 		\begin{bytefield}{32}
% 			\bitbox{5}{0x3}
% 			\bitbox{1}{\riscvbitboxaq}
% 			\bitbox{1}{\riscvbitboxrl}
% 			\bitbox{5}{rs2}
% 			\bitbox{5}{rs1}
% 			\bitbox{3}{0x1}
% 			\bitbox{5}{rd}
% 			\bitbox{7}{0x2f}
% 		\end{bytefield}
% 		\rvcheriasmfmt{\rvcheriasminsnnoref{SC.H} rd, rs2, rs1}

\optype{Address Construction}
The \asm{AUIPC} instruction is replaced by \asm{AUIPCC}, which derives capabilities from \PCC{}.
Our ABI also requires a new instruction, \asm{AUICGP}, that is similar to \asm{AUIPCC} but derives from \asm{\$c3} (\asm{\$cgp}).
This required allocating a new major opcode, although we expect that further support for linker relaxation may remove the need for \asm{AUICGP}. \\

	\begin{bytefield}{32}
	\bitheader[endianness=big]{0,6,7,11,12,31}\\
	\bitbox{20}{imm[31:12]}
	\bitbox{5}{cd}
	\bitbox{7}{0x17}
	\end{bytefield}
	\rvcheriasmfmt{\rvcheriasminsnref{AUIPCC} cd, imm}


	\begin{bytefield}{32}
		\bitbox{20}{imm[31:12]}
		\bitbox{5}{cd}
		\bitbox{7}{0x7b}
	\end{bytefield}
	\rvcheriasmfmt{\rvcheriasminsnref{AUICGP} cd, imm}

% 	\section{Assembly Programming}

% 	\subsection{Capability Register ABI Names}

% 	Table~\ref{table:riscv-register-names} lists the ABI names of
% 	the capability registers.  The ABI names follow from the ABI
% 	names of the RISC-V \textbf{x} registers.  All capability registers are
% 	Caller-Save in the hybrid ABI.	Capability registers follow
% 	the same save requirements as \textbf{x} registers in the purecap ABI.

% \begin{table}[h]
% \begin{center}
% \begin{tabular}{lllll}
% \toprule
% Register & ABI Name & Description						& Hybrid Saver & Purecap Saver \\
% \midrule
% c0	& cnull		& NULL pointer					& -		& - \\
% c1	& cra		& Return address					& Caller	& Caller \\
% c2	& csp		& Stack pointer					& Caller	& Callee \\
% c3	& cgp		& Global pointer					& -		& - \\
% c4	& ctp		& Thread pointer					& -		& - \\
% c5	& ct0		& Temporary/alternate link register	& Caller	& Caller \\
% c6-7 & ct1-2		& Temporaries						& Caller	& Caller \\
% c8	& cs0/cfp	& Saved register/frame pointer		& Caller	& Callee \\
% c9	& cs1		& Saved register					& Caller	& Callee \\
% c10-11 & ca0-1	& Function arguments/return values	& Caller	& Caller \\
% c12-17 & ca2-7	& Function arguments				& Caller	& Caller \\
% c18-27 & cs2-11	& Saved registers					& Caller	& Callee \\
% c28-31 & ct3-6	& Temporaries						& Caller	& Caller \\
% \bottomrule
% \end{tabular}
% \end{center}
% \caption{Assembler mnemonics for CHERI RISC-V capability registers}
% \label{table:riscv-register-names}
% \end{table}

% 	\subsection{Capability Encoding Mode Instructions}

% 	Table~\ref{table:riscv-capmode-instructions} lists uncompressed
% 	instructions which change semantics under capability mode.
% 	Table~\ref{table:riscv-capmode-instructions-rvc} lists compressed
% 	instructions which change semantics under capability mode.

% \begin{table}
% \begin{center}
% \begin{tabular}{ll}
% \toprule
% Integer Instruction		& Capability Instruction \\
% \midrule
% \texttt{l\{b|h|w|d\}[u] rd, offset(rs1)} & \texttt{cl\{b|h|w|d\}[u] rd, offset(cs1)} \\
% \texttt{lc cd, offset(rs1)} & \texttt{clc cd, offset(cs1)} \\
% \texttt{s\{b|h|w|d\} rs2, offset(rs1)} & \texttt{cs\{b|h|w|d\} rs2, offset(cs1)} \\
% \texttt{sc rs2, offset(rs1)} & \texttt{csc cs2, offset(cs1)} \\
% \texttt{fl\{h|w|d|q\} fd, offset(rs1)} & \texttt{cfl\{h|w|d|q\} fd, offset(cs1)} \\
% \texttt{fs\{h|w|d|q\} fs2, offset(rs1)} & \texttt{cfs\{h|w|d|q\} fs2, offset(cs1)} \\
% \texttt{lr.\{b|h|w|d\} rd, (rs1)} & \texttt{clr.\{b|h|w|d\} rd, (cs1)} \\
% \texttt{lr.c cd, (rs1)} & \texttt{clr.c cd, (cs1)} \\
% \texttt{sc.\{b|h|w|d\} rd, rs2, (rs1)} & \texttt{csc.\{b|h|w|d\} rd, rs2, (cs1)} \\
% \texttt{sc.c cd, cs2, (rs1)} & \texttt{csc.c cd, cs2, (cs1)} \\
% \texttt{amo<op>.\{w|d\}[.order] rd, rs2, (rs1)} & \texttt{camo<op>.\{w|d\}[.order] rd, rs2, (cs1)} \\
% \texttt{amo<op>.c[.order] cd, cs2, (rs1)} & \texttt{camo<op>.c[.order] cd, cs2, (cs1)} \\
% \texttt{auipc rd, offset} & \texttt{auipcc cd, offset} \\
% \bottomrule
% \end{tabular}
% \end{center}
% \caption{Uncompressed Instructions Dependent on Encoding Mode}
% \label{table:riscv-capmode-instructions}
% \end{table}

% \begin{table}
% \begin{center}
% \begin{tabular}{lll}
% \toprule
% Integer Instruction		& Capability Instruction & ISA \\
% \midrule
% \texttt{c.jr rs1} & \texttt{c.cjr cs1} & - \\
% \texttt{c.jalr rs1} & \texttt{c.cjalr cs1} & - \\
% \texttt{c.l\{w|d\} rd, offset(rs1)} & \texttt{c.cl\{w|d\} rd, offset(cs1)} & - \\
% \texttt{c.l\{w|d\}sp rd, offset(sp)} & \texttt{c.cl\{w|d\}sp rd, offset(csp)} & - \\
% \texttt{c.s\{w|d\} rs2, offset(rs1)} & \texttt{c.cs\{w|d\} rs2, offset(cs1)} & - \\
% \texttt{c.s\{w|d\}sp rs2, offset(sp)} & \texttt{c.cs\{w|d\}sp rs2, offset(csp)} & - \\
% \texttt{c.flw fd, offset(rs1)} & \texttt{c.clc cd, offset(cs1)} & RV32 \\
% \texttt{c.flwsp fd, offset(sp)} & \texttt{c.clcsp cd, offset(csp)} & RV32 \\
% \texttt{c.fsw fs2, offset(rs1)} & \texttt{c.csc cs2, offset(cs1)} & RV32 \\
% \texttt{c.fswsp fs2, offset(sp)} & \texttt{c.cscsp cs2, offset(csp)} & RV32 \\
% \texttt{c.fld fd, offset(rs1)} & \texttt{c.cfld fd, offset(cs1)} & RV32 \\
% \texttt{c.fldsp fd, offset(sp)} & \texttt{c.cfldsp fd, offset(csp)} & RV32 \\
% \texttt{c.fsd fs2, offset(rs1)} & \texttt{c.cfsd fs, offset(cs1)} & RV32 \\
% \texttt{c.fsdsp fs2, offset(sp)} & \texttt{c.cfsdsp fs, offset(csp)} & RV32 \\
% \texttt{c.fld fd, offset(rs1)} & \texttt{c.clc cd, offset(cs1)} & RV64 \\
% \texttt{c.fldsp fd, offset(sp)} & \texttt{c.clcsp cd, offset(csp)} & RV64 \\
% \texttt{c.fsd fs2, offset(rs1)} & \texttt{c.csc cs, offset(cs1)} & RV64 \\
% \texttt{c.fsdsp fs2, offset(sp)} & \texttt{c.cscsp cs, offset(csp)} & RV64 \\
% \bottomrule
% \end{tabular}
% \end{center}
% \caption{Compressed Instructions Dependent on Encoding Mode}
% \label{table:riscv-capmode-instructions-rvc}
% \end{table}

% 	Table~\ref{table:riscv-capmode-pseudo-remove} lists psuedoinstructions
% 	removed in capability mode.
% 	Table~\ref{table:riscv-capmode-pseudo-add} lists psuedoinstructions
% 	added in capability mode.

% \begin{table}
% \begin{center}
% \begin{tabular}{ll}
% \toprule
% Pseudoinstruction	& Meaning \\
% \midrule
% \texttt{la rd, symbol} & Load address \\
% \texttt{lla rd, symbol} & Load local address \\
% \texttt{l\{b|h|w|d\} rd, symbol} & Load global \\
% \texttt{s\{b|h|w|d\} rd, symbol, rt} & Store global \\
% \texttt{fl\{w|d\} rd, symbol, rt} & Floating-point load global \\
% \texttt{fs\{w|d\} rd, symbol, rt} & Floating-point store global \\
% \midrule
% \texttt{call symbol} & Call far-away subroutine \\
% \texttt{tail symbol} & Tail call far-away subroutine \\
% \bottomrule
% \end{tabular}
% \end{center}
% \caption{Pseudoinstructions Removed in Capability Mode}
% \label{table:riscv-capmode-pseudo-remove}
% \end{table}

% \begin{sidewaystable}
% \begin{center}
% \begin{tabular}{lll}
% \toprule
% Pseudoinstruction	& Base Instruction(s)	& Meaning \\
% \midrule
% \texttt{clgc cd, sym} &
%   \begin{tabular}{@{}l@{}}
%   \texttt{1: auipcc cd, \%captab\_pcrel\_hi(sym)} \\ \texttt{\ \ \ \ clc cd, \%pcrel\_lo(1b)(cd)}
%   \end{tabular}
%  & Load from capability table \\
% \texttt{cllc cd, sym} &
%   \begin{tabular}{@{}l@{}}
%   \texttt{1: auipcc cd, \%pcrel\_hi(sym)} \\ \texttt{\ \ \ \ cincoffset cd, cd, \%pcrel\_lo(1b)}
%   \end{tabular}
%  & Load PCC-relative capability \\
% \midrule
% \texttt{cjr cs} & \texttt{cjalr cnull, cs} & Jump to capability \\
% \texttt{cjalr cs} & \texttt{cjalr cra, cs} & Jump to capability and link \\
% \texttt{cret} & \texttt{cjalr cnull, cra} & Return to capability \\
% \midrule
% \texttt{cspecialr cd, scr} & \texttt{cspecialrw cd, scr, cnull} & Read special capability register \\
% \texttt{cspecialw scr, cs} & \texttt{cspecialrw cnull, scr, cs} & Write special capability register \\
% \bottomrule
% \end{tabular}
% \end{center}
% \caption{Pseudoinstructions Added in Capability Mode}
% \label{table:riscv-capmode-pseudo-add}
% % TODO: should the hyperrefs for these pseudos link to CJALR instead?
% \insnriscvlabel{cjr}
% \insnriscvlabel{cret}
% \insnriscvlabel{cspecialr}
% \insnriscvlabel{cspecialw}
% \insnriscvlabel{cllc}
% \insnriscvlabel{clgc}
% \end{sidewaystable}

	\section{Encoding Summary}

	The \cherimcuisa{} shares encodings with CHERI-RISC-V.
	The general-purpose instructions use the 0x5b major opcode and use the RISC-V R-type or I-type encoding formats.
	CHERI-RISC-V uses the funct3 field from bits 14-12 as a top-level opcode, and funct7 as a secondary
	opcode for standard 3-register operand instructions.
	Two-register operand instructions and single-register operand instructions are a subset
	of the 3-register operand encodings.

	\subsection*{Top-level encoding allocation (funct3 field)}
	{\rvcherienctablefontsize
	\rvcherienctabletop
	}

	\subsection*{Two Source \& Dest encoding allocation (funct7 field)}
	All three-register-operand (two sources, one destination) CHERI-RISC-V instructions use the RISC-V R-type encoding format, with the same funct field stored in funct7 and a 0 value in funct3.

	\vspace{1em}

	\rvcherirawbitbox{srcsrcdest}{func}{cd}{cs1}{rs2/cs2}

	\vspace{1em}

	{\rvcherienctablefontsize
	\def\rvcherireservedfootnotemark{$^\dagger$}
	\rvcherienctablesrcsrcdest\\\\
	\footnotesize
	$^\dagger$Reserved for future use.
	}

	\clearpage
	\subsection*{Two Source encoding allocation  (rd field)}
	There are currently no two source instructions but they would be of the following form:
	\vspace{1em}

	\rvcheriheader
	\rvcherirawbitbox{srcsrc}{func}{rs1/cs1}{rs2/cs2}

	\vspace{1em}

	{\rvcherienctablefontsize
	\def\rvcherireservedfootnotemark{$^\dagger$}
	\rvcherienctablesrcsrc\\\\
	\footnotesize
	$^\dagger$Reserved for future use.
	}

	\vspace{1em}

	\subsection*{One Source encoding allocation (rs2 field)}
	There are currently no one source instructions but they would be of the following form:

	\vspace{1em}

	\rvcheriheader
	\rvcherirawbitbox{src}{func}{rs1/cs1}

	\vspace{1em}

	{\rvcherienctablefontsize
	\def\rvcherireservedfootnotemark{$^\dagger$}
	\rvcherienctablesrc\\\\
	\footnotesize
	$^\dagger$Reserved for future use.
	}

	\vspace{1em}

	\subsection*{Source \& Dest encoding allocation (rs2 field)}
	Source \& Dest instructions are of the following form:

	\vspace{1em}

	\rvcheriheader
	\rvcherirawbitbox{srcdest}{func}{rd/cd}{rs1/cs1}

	\vspace{1em}

	{\rvcherienctablefontsize
	\def\rvcherireservedfootnotemark{$^\dagger$}
	\rvcherienctablesrcdest\\\\
	\footnotesize
	$^\dagger$Reserved for future use.
	}

	\vspace{1em}

	\subsection*{Dest-Only encoding allocation (rs1 field)}
	We do not currently have any one-register-operand instructions, but any
	future dest-only instructions will be of the following form:

	\vspace{1em}

	\rvcheriheader
	\rvcherirawbitbox{dest}{func}{rd}

	\vspace{1em}

	{\rvcherienctablefontsize
	\rvcherienctabledest
	}
